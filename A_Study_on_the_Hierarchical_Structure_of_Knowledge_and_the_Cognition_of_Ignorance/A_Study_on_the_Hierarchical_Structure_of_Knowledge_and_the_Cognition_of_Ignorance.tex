% Options for packages loaded elsewhere
\PassOptionsToPackage{unicode}{hyperref}
\PassOptionsToPackage{hyphens}{url}
%
\documentclass[
]{article}
\usepackage{amsmath,amssymb}
\usepackage{iftex}
\ifPDFTeX
  \usepackage[T1]{fontenc}
  \usepackage[utf8]{inputenc}
  \usepackage{textcomp} % provide euro and other symbols
\else % if luatex or xetex
  \usepackage{unicode-math} % this also loads fontspec
  \defaultfontfeatures{Scale=MatchLowercase}
  \defaultfontfeatures[\rmfamily]{Ligatures=TeX,Scale=1}
\fi
\usepackage{lmodern}
\ifPDFTeX\else
  % xetex/luatex font selection
\fi
% Use upquote if available, for straight quotes in verbatim environments
\IfFileExists{upquote.sty}{\usepackage{upquote}}{}
\IfFileExists{microtype.sty}{% use microtype if available
  \usepackage[]{microtype}
  \UseMicrotypeSet[protrusion]{basicmath} % disable protrusion for tt fonts
}{}
\makeatletter
\@ifundefined{KOMAClassName}{% if non-KOMA class
  \IfFileExists{parskip.sty}{%
    \usepackage{parskip}
  }{% else
    \setlength{\parindent}{0pt}
    \setlength{\parskip}{6pt plus 2pt minus 1pt}}
}{% if KOMA class
  \KOMAoptions{parskip=half}}
\makeatother
\usepackage{xcolor}
\usepackage{longtable,booktabs,array}
\usepackage{calc} % for calculating minipage widths
% Correct order of tables after \paragraph or \subparagraph
\usepackage{etoolbox}
\makeatletter
\patchcmd\longtable{\par}{\if@noskipsec\mbox{}\fi\par}{}{}
\makeatother
% Allow footnotes in longtable head/foot
\IfFileExists{footnotehyper.sty}{\usepackage{footnotehyper}}{\usepackage{footnote}}
\makesavenoteenv{longtable}
\setlength{\emergencystretch}{3em} % prevent overfull lines
\providecommand{\tightlist}{%
  \setlength{\itemsep}{0pt}\setlength{\parskip}{0pt}}
\setcounter{secnumdepth}{-\maxdimen} % remove section numbering
\ifLuaTeX
  \usepackage{selnolig}  % disable illegal ligatures
\fi
\IfFileExists{bookmark.sty}{\usepackage{bookmark}}{\usepackage{hyperref}}
\IfFileExists{xurl.sty}{\usepackage{xurl}}{} % add URL line breaks if available
\urlstyle{same}
\hypersetup{
  hidelinks,
  pdfcreator={LaTeX via pandoc}}

\author{}
\date{}

\begin{document}

\hypertarget{a-study-on-the-hierarchical-structure-of-knowledge-and-the-cognition-of-ignorance}{%
\section{A Study on the Hierarchical Structure of Knowledge and the
Cognition of
Ignorance}\label{a-study-on-the-hierarchical-structure-of-knowledge-and-the-cognition-of-ignorance}}

\emph{Kunihiro Sugiyama}\\
kunihiros@gmail.com

\hypertarget{introduction}{%
\subsection{Introduction}\label{introduction}}

Human knowledge possesses a multi-layered structure, and the ability to
recognize one's own ignorance (metacognition) is crucial for learning
and decision-making. This study proposes a recursive model of knowledge
and ignorance based on a single core function: \textbf{\(K\)}, which
represents epistemic recognition. By applying this function
recursively---\(K(x)\), \(K(K(x))\), \(K(K(K(x)))\), and so on---we
formalize the hierarchical structure of self-awareness that
distinguishes \textbf{Socratic wisdom} (``knowing that one does not
know'') from the \textbf{Dunning-Kruger effect} (``not knowing that one
does not know'').

Kant (1781), in his \emph{Critique of Pure Reason}, posed the
foundational question of epistemology: what are the limits of human
cognition, and can reason examine itself? His answer---that reason must
critique reason---established the recursive structure of self-reflection
as a philosophical problem. Yet Kant's contribution was
\textbf{descriptive}: he demonstrated that limits exist, but provided no
apparatus for locating their precise coordinates or guiding their
correction.

This study provides what classical epistemology could not: a
\textbf{mathematical framework} that transforms the Kantian question
from philosophical meditation into \textbf{operational methodology}. The
function \(K\) does not merely describe where cognition fails---it
provides the coordinates for \textbf{targeted intervention}. By
representing epistemic states on a continuous scale, we gain the
capacity to \textbf{observe the phenomenon of intelligence itself, and
to intervene in its structure}. This framework opens the possibility of
not only understanding cognition but \textbf{actively shaping its
trajectory toward new forms of knowing}.

This model integrates insights from \textbf{metacognition research},
\textbf{epistemology}, and \textbf{type theory} to address three aspects
that have not been sufficiently unified in existing research:

\begin{enumerate}
\def\labelenumi{\arabic{enumi}.}
\tightlist
\item
  The \textbf{recursive nature of self-awareness}: The same epistemic
  question (``Do I know?'') can be applied at every level of reflection.
\item
  The \textbf{continuous gradation of knowledge}: Knowledge states exist
  on a continuum from complete misconception (\(-1\)) through ignorance
  (\(0\)) to accurate knowledge (\(1\)).
\item
  The \textbf{distinction between epistemic state and phenomenological
  confidence}: What one knows versus how certain one feels are
  orthogonal dimensions.
\end{enumerate}

By presenting a mathematically rigorous yet philosophically grounded
framework, this study seeks to deepen our understanding of the structure
of knowledge and the cognitive mechanisms of ignorance.

\hypertarget{contribution-a-conceptual-foundation}{%
\subsubsection{Contribution: A Conceptual
Foundation}\label{contribution-a-conceptual-foundation}}

This paper establishes the \textbf{conceptual foundation} for a unified
theory of recursive metacognition. The trichotomy of epistemic
states---\textbf{knowing}, \textbf{not knowing}, and
\textbf{misunderstanding}---is a universal human experience that
transcends cultures, domains, and disciplines. Before elaborating
measurement-theoretic models or conducting empirical validation, we must
first \textbf{settle the conceptual vocabulary}.

\textbf{What this paper provides:} 1. A \textbf{single, unified operator
\(K\)} that applies recursively at all levels of self-reflection 2. A
\textbf{purely observational framework} where \(K\) is not a mental
process but an \textbf{observation protocol} 3. A \textbf{complete
taxonomy} (27 patterns) that classifies all possible metacognitive
configurations 4. A \textbf{resolution of apparent contradictions}
(e.g., \(K(0) = 0\) vs \(K_1 = -1\)) through layer independence

\textbf{What this paper deliberately does not provide:} - Probabilistic
measurement models (future work) - Empirical validation (orthogonal
contribution) - Formal type-theoretic proofs (analogical treatment
suffices for conceptual clarity)

These omissions are not gaps but \textbf{scope boundaries}.
Measurement-theoretic elaboration and empirical validation require this
conceptual foundation to be settled first. We invite the research
community to build upon this foundation.

\hypertarget{philosophical-foundation-and-interpretive-notes}{%
\subsection{Philosophical Foundation and Interpretive
Notes}\label{philosophical-foundation-and-interpretive-notes}}

This section clarifies the philosophical motivation behind this paper
and provides essential interpretive guidance to prevent misunderstanding
of the proposed model.

\hypertarget{theoretical-rationale}{%
\subsubsection{Theoretical Rationale}\label{theoretical-rationale}}

This study is grounded in the logical structure of recursive ignorance,
exemplified by the proposition \textbf{``I don't know what I don't
know.''} If ``knowing one's ignorance'' (Socratic wisdom) is a
recognized concept, then logically, ``not knowing one's ignorance'' must
also exist. And if that exists, then so must ``not knowing that one
doesn't know one's ignorance''---and so on, recursively.

The goal of this paper is to \textbf{mathematically formalize this
recursive structure of knowledge and ignorance}, not to judge or rank
cognitive states.

\hypertarget{descriptive-nature-of-the-scale}{%
\subsubsection{Descriptive Nature of the
Scale}\label{descriptive-nature-of-the-scale}}

The values \(-1\), \(0\), and \(1\) in this model function as
\textbf{epistemic state descriptors}. They serve as epistemic
coordinates rather than normative metrics (e.g., ``good'' or ``bad'').

\begin{longtable}[]{@{}cl@{}}
\toprule\noalign{}
Value & Meaning \\
\midrule\noalign{}
\endhead
\bottomrule\noalign{}
\endlastfoot
\(1\) & The subject holds correct knowledge. \\
\(0\) & The subject lacks knowledge (ignorance). \\
\(-1\) & The subject holds incorrect knowledge (misconception). \\
\end{longtable}

A subject in state \(-1\) (misconception) is not normatively inferior to
a subject in state \(0\) (ignorance); they occupy \textbf{distinct
epistemic loci}. Whether one state is ``preferable'' to another depends
on context, goals, and values---domains outside the scope of this model.

\hypertarget{separation-of-knowledge-and-confidence}{%
\subsubsection{Separation of Knowledge and
Confidence}\label{separation-of-knowledge-and-confidence}}

A fundamental distinction in this framework is that \textbf{the function
\(K\) measures epistemic state, not phenomenological confidence}.
Confidence is a separate dimension that will be introduced later in the
measurement section.

\begin{itemize}
\tightlist
\item
  \(K(x)\): How accurately the subject recognizes object \(x\)
  (epistemic state).
\item
  \(C\) (Confidence): How certain the subject feels about their
  recognition (phenomenological experience).
\end{itemize}

This separation is essential for capturing phenomena like the
Dunning-Kruger effect, where \(K(x) = 0\) (the subject does not know)
but \(K(K(x)) = -1\) (the subject misrecognizes their ignorance), often
accompanied by high subjective confidence.

\hypertarget{scope-clarification-methodological-relativism}{%
\subsection{Scope Clarification: Methodological
Relativism}\label{scope-clarification-methodological-relativism}}

This section clarifies the scope and philosophical stance of this
framework.

\hypertarget{what-this-model-does}{%
\subsubsection{What This Model Does}\label{what-this-model-does}}

This model provides a \textbf{mathematical apparatus} for representing
and manipulating the \textbf{structure of epistemic states} relative to
a proposition. The function \(K(x)\) measures the subject's epistemic
state regarding proposition \(x\):

\begin{longtable}[]{@{}cll@{}}
\toprule\noalign{}
\(K(x)\) & State & Interpretation \\
\midrule\noalign{}
\endhead
\bottomrule\noalign{}
\endlastfoot
\(1\) & Aligned & Subject's belief is consistent with \(x\) \\
\(0\) & Indeterminate & Subject has no determinate stance on \(x\) \\
\(-1\) & Opposed & Subject's belief is contrary to \(x\) \\
\end{longtable}

\hypertarget{what-this-model-does-not-do}{%
\subsubsection{What This Model Does Not
Do}\label{what-this-model-does-not-do}}

This model \textbf{does not adjudicate} what is ``correct'' or ``true.''
The designation of a proposition as the ``target'' (such that
\(K(x) = 1\) represents success) is a \textbf{methodological choice}
made by the experimenter, not a claim of this framework.

The proposition \(x\) itself serves as the \textbf{implicit reference
point}. What counts as ``aligned'' (\(K(x) = 1\)) versus ``opposed''
(\(K(x) = -1\)) is determined by the \textbf{experimental context}
(e.g., expert consensus, empirical measurement, community agreement),
not by this model.

\hypertarget{methodological-relativism}{%
\subsubsection{Methodological
Relativism}\label{methodological-relativism}}

This framework adopts a position of \textbf{methodological relativism}:
the reference point is necessarily context-dependent, and the model
operates on the structure of epistemic states relative to that chosen
reference. This design allows researchers with different philosophical
commitments (realism, relativism, pragmatism) to use the same
mathematical framework while maintaining their preferred interpretation
of what constitutes ``correct'' knowledge.

\hypertarget{the-recursive-structure-kkkx}{%
\subsection{\texorpdfstring{The Recursive Structure:
\(K(K(K(x)))\)}{The Recursive Structure: K(K(K(x)))}}\label{the-recursive-structure-kkkx}}

The cognitive structure of knowledge is modeled using a
\textbf{recursive epistemic function} \(K\). This model formalizes the
intuition that the same question---``Do I know?''---can be applied at
every level of self-reflection.

\hypertarget{formal-definition-of-k}{%
\subsubsection{\texorpdfstring{Formal Definition of
\(K\)}{Formal Definition of K}}\label{formal-definition-of-k}}

We adopt an \textbf{observational family} interpretation of \(K\), which
provides a clear and consistent framework for understanding recursive
metacognition.

\hypertarget{formal-framework-layered-observation-model}{%
\paragraph{Formal Framework: Layered Observation
Model}\label{formal-framework-layered-observation-model}}

\textbf{Definition (Observation Family):}

Let \(\{K^{(n)}\}_{n=0}^{\infty}\) be a family of observation functions,
where each \(K^{(n)}\) maps from a layer-specific state space to the
epistemic scale \([-1, 1]\):

\[K^{(n)}: \mathcal{S}_n \to [-1, 1]\]

\textbf{Definition (State Hierarchy):}

\begin{itemize}
\tightlist
\item
  \(\mathcal{S}_0\): First-order epistemic states (correctness of
  responses)
\item
  \(\mathcal{S}_1\): Metacognitive states (alignment between claims and
  \(\mathcal{S}_0\))
\item
  \(\mathcal{S}_n\): n-th order states (alignment between claims and
  \(\mathcal{S}_{n-1}\))
\end{itemize}

\textbf{Notational Convention:}

The notation \(K(K(x))\) is a \textbf{shorthand} for
\(K^{(1)}(\text{State}_1(x))\), not numerical composition.

More precisely: - \(K_0(x) := K^{(0)}(\text{State}_0(x))\) -
\(K_1(x) := K^{(1)}(\text{State}_1(x))\) -
\(K_n(x) := K^{(n)}(\text{State}_n(x))\)

\textbf{Shared Anchor Semantics:}

All \(K^{(n)}\) share the same anchor constraints: -
\(K^{(n)}(\text{"knowledge"}) = 1\) -
\(K^{(n)}(\text{"ignorance"}) = 0\) -
\(K^{(n)}(\text{"misconception"}) = -1\)

This ensures cross-layer comparability while allowing distinct
measurement procedures per layer.

\hypertarget{entry-and-recursive-mappings}{%
\paragraph{Entry and Recursive
Mappings}\label{entry-and-recursive-mappings}}

\textbf{1. Entry Mapping (Layer 0):}

For an abstract object \(x\) (e.g., a proposition, a task item, or any
epistemic target), the subject's first-order epistemic condition is
represented as a continuous value:

\[k_0(x) \in [-1, 1]\]

This is the \textbf{only point} where the external object \(x\) enters
the model. The internal representation \(k_0\) captures how the subject
stands with respect to \(x\).

\textbf{2. Recursive Mapping (Layers \(n \ge 1\)):}

At all higher layers, the observation family \(\{K^{(n)}\}\) acts on
layer-specific states:

\[K^{(n)}: \mathcal{S}_n \to [-1, 1]\]

The ``object'' of higher-order \(K^{(n)}\) is not the numerical output
of the previous layer but the \textbf{distinct state object}
\(\text{State}_n\).

\textbf{Output Interpretation (Prototypical Anchor Points):}

The values \(-1\), \(0\), and \(1\) serve as \textbf{prototypical
anchors} on the continuous scale \([-1, 1]\):

\begin{itemize}
\tightlist
\item
  \(K(\cdot) = 1\): The subject accurately recognizes the target (full
  knowledge or accurate metacognition).
\item
  \(K(\cdot) = 0\): The subject has no determinate stance regarding the
  target (pure ignorance).
\item
  \(K(\cdot) = -1\): The subject misrecognizes the target (misconception
  or metacognitive failure).
\end{itemize}

All intermediate values represent \textbf{graded mixtures} of these
prototypes (partial knowledge, partial misconception, uncertainty,
etc.).

\textbf{Key Insight:} The function \(K\) has \textbf{consistent
semantics} across all layers: ``How accurately does the subject
recognize this target?'' At Layer 0, the target is an external object
\(x\). At Layers \(n \ge 1\), the target is the subject's own epistemic
state \(k_{n-1}\).

\hypertarget{observation-and-objects-the-purely-observational-framework}{%
\subsubsection{Observation and Objects: The Purely Observational
Framework}\label{observation-and-objects-the-purely-observational-framework}}

This framework adopts a \textbf{purely observational stance}: epistemic
states are operationalized as observable responses, and metacognitive
states as observable alignments between claims and performance. We do
not posit internal ``beliefs'' or ``perceptions'' --- only
\textbf{states} that are measurable by an external observer.

\textbf{The Observer:}

In experimental settings, the ``observer'' is the \textbf{experimenter}
who: 1. Presents a task/question 2. Records the respondent's answer 3.
Compares the answer to a reference 4. Assigns a \(K\) value based on the
comparison

The Observer does not access internal ``perception'' or ``belief.'' The
Observer only sees \textbf{observable behavior}: answers, claims,
responses.

\textbf{Formal Definition of State\(_n\):}

\textbf{State\(_0\) (First-Order Epistemic State):}

\[\text{State}_0(x) = f_0(\text{Response}(x), \text{Reference}(x))\]

Where: - \(\text{Response}(x)\): Subject's answer to item \(x\) -
\(\text{Reference}(x)\): Ground truth or expert consensus - \(f_0\):
Comparison function yielding \{correct, incorrect, absent\}

\textbf{State\(_1\) (Metacognitive State):}

\[\text{State}_1(x) = f_1(\text{Claim}_1(x), \text{State}_0(x))\]

Where: - \(\text{Claim}_1(x)\): Subject's metacognitive claim (``I
know'' / ``I don't know'' / ``I'm wrong'') - \(f_1\): Alignment function
yielding \{aligned, uncertain, misaligned\}

\textbf{State\(_n\) (n-th Order State):}

\[\text{State}_n(x) = f_n(\text{Claim}_n(x), \text{State}_{n-1}(x))\]

\textbf{Graphical Model:}

\begin{verbatim}
Response(x) ──┐
              ├─→ f_0 ──→ State_0 ──→ K^(0) ──→ K_0
Reference(x) ─┘              │
                             ↓
Claim_1(x) ─────────────→ f_1 ──→ State_1 ──→ K^(1) ──→ K_1
                                    │
                                    ↓
Claim_2(x) ─────────────────────→ f_2 ──→ State_2 ──→ K^(2) ──→ K_2
\end{verbatim}

\textbf{Operational Interpretation:}

\begin{itemize}
\tightlist
\item
  \textbf{State\(_0\) (first-order epistemic state)}: The respondent's
  answer compared to a reference.

  \begin{itemize}
  \tightlist
  \item
    Operationalized as: ``Is the answer correct, incorrect, or absent?''
  \end{itemize}
\item
  \textbf{State\(_1\) (metacognitive state)}: The alignment between the
  respondent's metacognitive claim and their actual State\(_0\).

  \begin{itemize}
  \tightlist
  \item
    Operationalized as: ``Does the claim `I know' match the actual
    correctness?''
  \end{itemize}
\item
  \textbf{State\(_2\) (meta-metacognitive state)}: The alignment between
  the respondent's meta-metacognitive claim and their actual
  State\(_1\).
\end{itemize}

\textbf{Example:} - Respondent answers incorrectly → State\(_0\) =
``ignorance'' → \(K_0 = 0\) - Respondent claims ``I know'' → State\(_1\)
= ``misalignment'' → \(K_1 = -1\) - Respondent claims ``My
self-assessment is accurate'' → State\(_2\) = ``misalignment'' →
\(K_2 = -1\)

\textbf{Critical Clarification:}

Each \(K_n\) is an \textbf{independent observation} of a
\textbf{distinct object} (State\(_n\)):

\begin{itemize}
\tightlist
\item
  \textbf{\(K_0(x)\)}: Observer's measurement of \textbf{State\(_0\)}
\item
  \textbf{\(K_1(x)\)}: Observer's measurement of \textbf{State\(_1\)}
\item
  \textbf{\(K_2(x)\)}: Observer's measurement of \textbf{State\(_2\)}
\end{itemize}

\[K_1(x) \neq K(K_0(x))\]

\(K_1\) is \textbf{not} ``applying \(K\) to the numerical value of
\(K_0\).'' \(K_1\) is ``observing a different object (State\(_1\)) and
reporting the measurement.''

\textbf{Resolving the Apparent Contradiction (\(K(0) = 0\) vs
\(K_1 = -1\)):}

The axiom \(K(0) = 0\) means: ``If the observed state is `ignorance'
(0), the measurement result is `ignorance' (0).''

In the Dunning-Kruger case: - \(K_0(x) = 0\): Observer measures
State\(_0\) → ``ignorance'' - \(K_1(x) = -1\): Observer measures
State\(_1\) → ``misalignment''

\textbf{State\(_1\) is not ``0''.} State\(_1\) is the metacognitive
state (alignment/misalignment), which the observer measures as
``misrecognition'' (-1).

The axiom \(K(0) = 0\) does not apply because the input to \(K_1\) is
\textbf{not} the number ``0''. The input is \textbf{State\(_1\)}, a
different object entirely.

\textbf{Analogy (Thermometer Calibration):}

Consider a thermometer and its calibration: - \textbf{State\(_0\)
(temperature)}: The actual temperature of water = 20°C -
\textbf{State\(_1\) (thermometer accuracy)}: Whether the thermometer
correctly reads State\(_0\)

Measuring State\(_0\) = 20°C does not constrain State\(_1\). The
thermometer might be: - Accurate (State\(_1\) = correct) → \(K_1 = 1\) -
Miscalibrated (State\(_1\) = incorrect) → \(K_1 = -1\)

\(K_1\) measures a \textbf{property of the measuring instrument}, not
the original object. Similarly, \(K_1\) measures the accuracy of
\textbf{the respondent's self-monitoring}, not the first-order state
itself.

\hypertarget{type-theoretic-foundation}{%
\subsubsection{Type-Theoretic
Foundation}\label{type-theoretic-foundation}}

To address concerns about mathematical rigor, we provide a
type-theoretic justification for the recursive structure.

\textbf{Core Principle:} All epistemic states live on a \textbf{single
continuous scale} \([-1, 1]\). The recursive structure is well-defined
because \(K\) maps this space to itself.

\textbf{Two-Stage Type Structure:}

\begin{verbatim}
-- Layer 0: Entry from abstract object to epistemic state
k0 : Object -> Real[-1, 1]

-- Layers n >= 1: Recursive self-application on the epistemic state space
K  : Real[-1, 1] -> Real[-1, 1]
\end{verbatim}

\textbf{Recursive Application:}

Once we are on the epistemic state space \([-1, 1]\), recursion is
straightforward:

\begin{verbatim}
k0 = k0(x)           -- Entry: external object -> epistemic state
k1 = K(k0)           -- Layer 1: metacognition of k0
k2 = K(k1)           -- Layer 2: metacognition of k1
...
\end{verbatim}

\textbf{This is not a type error.} This is a \textbf{recursive type}
with a well-defined structure, analogous to: - \textbf{Lambda Calculus}:
Self-application via \(\lambda x. (\lambda y. y) x\) -
\textbf{Fixed-Point Combinators}:
\(Y = \lambda f. (\lambda x. f(x x))(\lambda x. f(x x))\) -
\textbf{Recursive Types}: \(\mu \alpha. \alpha \to \alpha\)

The recursive structure \(K(K(x))\) is mathematically well-founded and
has precedent in formal systems.

\hypertarget{axiomatic-constraints-on-k}{%
\subsubsection{\texorpdfstring{Axiomatic Constraints on
\(K\)}{Axiomatic Constraints on K}}\label{axiomatic-constraints-on-k}}

We impose the following minimal constraints on the epistemic function
\(K\):

\textbf{Definition: Objective Evaluation}

The function \(K\) represents an \textbf{objective evaluation} of the
subject's epistemic state by an external observer (or the system),
distinct from the subject's subjective feeling of confidence (\(C\)).

\begin{itemize}
\tightlist
\item
  \(K(x) = 1\): Objectively accurate recognition.
\item
  \(K(x) = -1\): Objectively inverted recognition (misconception).
\end{itemize}

\textbf{Axiom Scope:}

The axioms describe the behavior of the observation function \(K\)
\textbf{within a single layer}.

\begin{itemize}
\tightlist
\item
  \(K(1) = 1\): If the observed state is ``knowledge'', report
  ``knowledge''
\item
  \(K(0) = 0\): If the observed state is ``ignorance'', report
  ``ignorance''
\item
  \(K(-1) = -1\): If the observed state is ``misconception'', report
  ``misconception''
\end{itemize}

\textbf{Layer Independence:}

Each \(K_n\) observes a \textbf{different object} (State\(_n\)). The
axioms apply to each observation independently.

\begin{longtable}[]{@{}
  >{\centering\arraybackslash}p{(\columnwidth - 4\tabcolsep) * \real{0.3846}}
  >{\raggedright\arraybackslash}p{(\columnwidth - 4\tabcolsep) * \real{0.3077}}
  >{\raggedright\arraybackslash}p{(\columnwidth - 4\tabcolsep) * \real{0.3077}}@{}}
\toprule\noalign{}
\begin{minipage}[b]{\linewidth}\centering
Layer
\end{minipage} & \begin{minipage}[b]{\linewidth}\raggedright
Object Observed
\end{minipage} & \begin{minipage}[b]{\linewidth}\raggedright
Axiom Application
\end{minipage} \\
\midrule\noalign{}
\endhead
\bottomrule\noalign{}
\endlastfoot
\(K_0\) & State\(_0\) (first-order epistemic state) & \(K\)(State\(_0\))
follows axioms \\
\(K_1\) & State\(_1\) (metacognitive state) & \(K\)(State\(_1\)) follows
axioms \\
\(K_2\) & State\(_2\) (meta-metacognitive state) & \(K\)(State\(_2\))
follows axioms \\
\end{longtable}

\textbf{Why \(K_0 = 0\) and \(K_1 = -1\) is NOT a contradiction:}

\begin{itemize}
\tightlist
\item
  \(K_0 = 0\): Observer measures State\(_0\) as ``ignorance''
\item
  \(K_1 = -1\): Observer measures State\(_1\) as ``misrecognition''
\end{itemize}

State\(_0 \neq\) State\(_1\). They are different objects. The axiom
\(K(0) = 0\) applies to State\(_0\), not to State\(_1\).

\textbf{Monotonicity:}

Monotonicity applies \textbf{within each layer}: if State \(>\) State',
then \(K\)(State) \(\geq\) \(K\)(State').

It does NOT constrain relationships \textbf{across layers} (e.g.,
\(K_0\) vs \(K_1\)).

\textbf{Boundedness:}

\(K: [-1, 1] \to [-1, 1]\)

Each observation is bounded on this interval.

\textbf{Note:} We deliberately refrain from specifying stronger
constraints (e.g., odd symmetry, Lipschitz constant, contraction
mapping) at this stage. The framework is intended to be
\textbf{descriptive} rather than \textbf{predictive}---it provides a
vocabulary for classifying observed metacognitive states, not a
generative model of metacognitive dynamics. Specifying a particular
functional form for \(K\) is a task for domain-specific empirical
research.

\hypertarget{recursive-application}{%
\subsubsection{Recursive Application}\label{recursive-application}}

\textbf{Notation:}

We use subscript notation to denote the layer of observation:

\[K_0(x) = K(\text{State}_0)\] \[K_1(x) = K(\text{State}_1)\]
\[K_2(x) = K(\text{State}_2)\]

\textbf{Important Clarification:}

The traditional notation \(K(K(x))\) is \textbf{shorthand} for
\(K_1(x)\), but it should \textbf{not} be interpreted as function
composition (i.e., \(K(K_0(x))\) where the numerical value of \(K_0\) is
passed to \(K\)).

Each \(K_n\) observes a \textbf{distinct object} (State\(_n\)), not the
numerical output of the previous layer.

\textbf{Interpretation:}

\begin{longtable}[]{@{}
  >{\raggedright\arraybackslash}p{(\columnwidth - 6\tabcolsep) * \real{0.2500}}
  >{\raggedright\arraybackslash}p{(\columnwidth - 6\tabcolsep) * \real{0.2500}}
  >{\raggedright\arraybackslash}p{(\columnwidth - 6\tabcolsep) * \real{0.2500}}
  >{\raggedright\arraybackslash}p{(\columnwidth - 6\tabcolsep) * \real{0.2500}}@{}}
\toprule\noalign{}
\begin{minipage}[b]{\linewidth}\raggedright
Layer
\end{minipage} & \begin{minipage}[b]{\linewidth}\raggedright
Expression
\end{minipage} & \begin{minipage}[b]{\linewidth}\raggedright
Object Observed
\end{minipage} & \begin{minipage}[b]{\linewidth}\raggedright
Question
\end{minipage} \\
\midrule\noalign{}
\endhead
\bottomrule\noalign{}
\endlastfoot
\textbf{Layer 0} & \(K_0(x)\) & State\(_0\) (first-order epistemic
state) & ``Is the respondent's answer correct?'' \\
\textbf{Layer 1} & \(K_1(x)\) & State\(_1\) (metacognitive state) & ``Is
the respondent's self-assessment aligned with their actual
State\(_0\)?'' \\
\textbf{Layer 2} & \(K_2(x)\) & State\(_2\) (meta-metacognitive state) &
``Is the respondent's meta-self-assessment aligned with their actual
State\(_1\)?'' \\
\end{longtable}

\textbf{Same observation function \(K\). Different objects
(State\(_n\)). Independent measurements.}

\hypertarget{examples}{%
\subsubsection{Examples}\label{examples}}

\hypertarget{example-1-knowing-knowledge}{%
\paragraph{Example 1: Knowing
Knowledge}\label{example-1-knowing-knowledge}}

\begin{itemize}
\tightlist
\item
  \(K(x) = 1\): The subject knows that ``water boils at 100°C.''
\item
  \(K(K(x)) = 1\): The subject accurately recognizes that they know this
  fact.
\item
  \textbf{Classification}: Knowing Knowledge (accurate self-awareness)
\end{itemize}

\hypertarget{example-2-socratic-wisdom}{%
\paragraph{Example 2: Socratic Wisdom}\label{example-2-socratic-wisdom}}

\begin{itemize}
\tightlist
\item
  \(K(x) = 0\): The subject does not know the boiling point of water.
\item
  \(K(K(x)) = 1\): The subject accurately recognizes their ignorance
  (``I know that I don't know'').
\item
  \textbf{Classification}: Knowing Ignorance (Socratic wisdom)
\end{itemize}

\hypertarget{example-3-dunning-kruger-effect}{%
\paragraph{Example 3: Dunning-Kruger
Effect}\label{example-3-dunning-kruger-effect}}

\begin{itemize}
\tightlist
\item
  \(K(x) = 0\): The subject does not know the boiling point of water.
\item
  \(K(K(x)) = -1\): The subject misrecognizes their ignorance, believing
  they know.
\item
  \textbf{Classification}: Unknowing Ignorance (Dunning-Kruger effect)
\end{itemize}

\hypertarget{example-4-imposter-syndrome}{%
\paragraph{Example 4: Imposter
Syndrome}\label{example-4-imposter-syndrome}}

\begin{itemize}
\tightlist
\item
  \(K(x) = 1\): The subject knows that ``water boils at 100°C.''
\item
  \(K(K(x)) = -1\) or \(0\): The subject does not recognize their
  knowledge (``I don't think I know this'').
\item
  \textbf{Classification}: Unknowing Knowledge (imposter syndrome)
\end{itemize}

\hypertarget{the-four-quadrants-of-metacognition}{%
\subsubsection{The Four Quadrants of
Metacognition}\label{the-four-quadrants-of-metacognition}}

The relationship between \(K(x)\) (actual state) and \(K(K(x))\)
(metacognitive accuracy) produces four archetypal patterns:

\begin{longtable}[]{@{}
  >{\centering\arraybackslash}p{(\columnwidth - 6\tabcolsep) * \real{0.2778}}
  >{\centering\arraybackslash}p{(\columnwidth - 6\tabcolsep) * \real{0.2778}}
  >{\raggedright\arraybackslash}p{(\columnwidth - 6\tabcolsep) * \real{0.2222}}
  >{\raggedright\arraybackslash}p{(\columnwidth - 6\tabcolsep) * \real{0.2222}}@{}}
\toprule\noalign{}
\begin{minipage}[b]{\linewidth}\centering
\(K(x)\)
\end{minipage} & \begin{minipage}[b]{\linewidth}\centering
\(K(K(x))\)
\end{minipage} & \begin{minipage}[b]{\linewidth}\raggedright
Classification
\end{minipage} & \begin{minipage}[b]{\linewidth}\raggedright
Interpretation
\end{minipage} \\
\midrule\noalign{}
\endhead
\bottomrule\noalign{}
\endlastfoot
\(1\) (Know) & \(1\) (Accurate) & \textbf{Knowing Knowledge} & Accurate
self-awareness \\
\(0\) (Ignorant) & \(1\) (Accurate) & \textbf{Knowing Ignorance} &
Socratic wisdom \\
\(0\) (Ignorant) & \(-1\) (Misrecognition) & \textbf{Unknowing
Ignorance} & Dunning-Kruger effect \\
\(1\) (Know) & \(-1\) or \(0\) & \textbf{Unknowing Knowledge} & Imposter
syndrome \\
\end{longtable}

\textbf{Important Note:} The value \(K(K(x)) = -1\) for ``Unknowing
Ignorance'' does \textbf{not} mean it is ``bad'' in a normative sense.
It simply describes an epistemic state where the subject
\textbf{misrecognizes their own ignorance}. Whether this is problematic
depends on context and goals.

\hypertarget{complete-taxonomy-kux2080-kux2081-kux2082-27-patterns}{%
\subsubsection{Complete Taxonomy: K₀ × K₁ × K₂ (27
Patterns)}\label{complete-taxonomy-kux2080-kux2081-kux2082-27-patterns}}

With three prototypical values \{-1, 0, 1\} at each of three layers,
there are 27 possible configurations. This taxonomy demonstrates the
theoretical value of higher-order reflection (\(K_2\)).

\hypertarget{kux2080-1-knowledge}{%
\paragraph{K₀ = 1 (Knowledge)}\label{kux2080-1-knowledge}}

\begin{longtable}[]{@{}cccl@{}}
\toprule\noalign{}
\(K_0\) & \(K_1\) & \(K_2\) & Interpretation \\
\midrule\noalign{}
\endhead
\bottomrule\noalign{}
\endlastfoot
1 & 1 & 1 & ✓ Perfect self-awareness (Ideal) \\
1 & 1 & 0 & Knowing Knowledge, uncertain about meta \\
1 & 1 & -1 & Knowing Knowledge, misrecognizes meta \\
1 & 0 & 1 & Uncertain about knowledge, recognizes uncertainty \\
1 & 0 & 0 & Generally uncertain \\
1 & 0 & -1 & Uncertain about knowledge, misrecognizes it \\
1 & -1 & 1 & \textbf{Imposter Syndrome (self-aware)} \\
1 & -1 & 0 & Imposter Syndrome (ambiguous) \\
1 & -1 & -1 & \textbf{Severe Imposter Syndrome} \\
\end{longtable}

\hypertarget{kux2080-0-ignorance}{%
\paragraph{K₀ = 0 (Ignorance)}\label{kux2080-0-ignorance}}

\begin{longtable}[]{@{}cccl@{}}
\toprule\noalign{}
\(K_0\) & \(K_1\) & \(K_2\) & Interpretation \\
\midrule\noalign{}
\endhead
\bottomrule\noalign{}
\endlastfoot
0 & 1 & 1 & ✓ \textbf{Perfect Socratic Wisdom} \\
0 & 1 & 0 & Socratic Wisdom, uncertain about meta \\
0 & 1 & -1 & Socratic Wisdom, misrecognizes meta \\
0 & 0 & 1 & Uncertain about ignorance, recognizes uncertainty \\
0 & 0 & 0 & Generally uncertain \\
0 & 0 & -1 & Uncertain about ignorance, misrecognizes it \\
0 & -1 & 1 & \textbf{Dunning-Kruger (self-aware)} --- can improve \\
0 & -1 & 0 & Dunning-Kruger (ambiguous) \\
0 & -1 & -1 & \textbf{Severe Dunning-Kruger} --- resistant to
correction \\
\end{longtable}

\hypertarget{kux2080--1-misconception}{%
\paragraph{K₀ = -1 (Misconception)}\label{kux2080--1-misconception}}

\begin{longtable}[]{@{}cccl@{}}
\toprule\noalign{}
\(K_0\) & \(K_1\) & \(K_2\) & Interpretation \\
\midrule\noalign{}
\endhead
\bottomrule\noalign{}
\endlastfoot
-1 & 1 & 1 & ✓ Self-aware misconception (rare, correctable) \\
-1 & 1 & 0 & Knowing Misconception, uncertain about meta \\
-1 & 1 & -1 & Knowing Misconception, misrecognizes meta \\
-1 & 0 & 1 & Uncertain about misconception, recognizes uncertainty \\
-1 & 0 & 0 & Generally uncertain \\
-1 & 0 & -1 & Uncertain about misconception, misrecognizes it \\
-1 & -1 & 1 & \textbf{Confident wrong (self-aware)} \\
-1 & -1 & 0 & Confident wrong (ambiguous) \\
-1 & -1 & -1 & \textbf{Most dangerous: Severe confident wrong} \\
\end{longtable}

\textbf{Theoretical Value of \(K_2\):}

The third layer (\(K_2\)) enables modeling of \textbf{metacognitive
interventions} and their effectiveness:

\begin{itemize}
\tightlist
\item
  \(K_2 = 1\) with \(K_1 = -1\): Subject recognizes their metacognitive
  failure → \textbf{teachable moment}
\item
  \(K_2 = -1\) with \(K_1 = -1\): Subject does not recognize their
  failure → \textbf{resistant to intervention}
\end{itemize}

Higher-order reflection (\(K_2\), \(K_3\), \ldots) provides diagnostic
power for identifying when and how metacognitive correction is possible.

\hypertarget{connection-with-metacognition-research}{%
\subsubsection{Connection with Metacognition
Research}\label{connection-with-metacognition-research}}

\textbf{Flavell (1979)} defined metacognition as ``the ability to
monitor and control one's own cognitive activities.'' The recursive
structure (\(K \to K(K) \to K(K(K))\)) formalizes this concept
mathematically.

\textbf{Nelson \& Narens (1990)} introduced the influential
\textbf{monitoring/control framework}, distinguishing between the
\textbf{object level} (cognitive processes) and the \textbf{meta level}
(monitoring and control of cognition). Our framework directly
corresponds to this structure:

\begin{longtable}[]{@{}ll@{}}
\toprule\noalign{}
Nelson \& Narens & This Framework \\
\midrule\noalign{}
\endhead
\bottomrule\noalign{}
\endlastfoot
Object level & State\(_0\) (first-order epistemic state) \\
Meta level (monitoring) & State\(_1\) (metacognitive state) \\
Control signal & Not modeled (orthogonal dimension) \\
\end{longtable}

Our \(K_0\) and \(K_1\) formalize the object/meta distinction with a
\textbf{single unified operator}, providing mathematical precision to
Nelson \& Narens' conceptual framework.

\textbf{Koriat (1993)} proposed the \textbf{cue-utilization theory},
explaining how confidence arises from accessibility and familiarity cues
rather than direct access to accuracy. This distinction between
cue-based confidence and actual accuracy corresponds precisely to our
separation of \(C\) (confidence) and \(K\) (epistemic state). Our
framework accommodates cue-based confidence as a component of \(C\),
while \(K\) measures the objective alignment between the subject's state
and reality.

\textbf{Kruger and Dunning (1999)} demonstrated that individuals with
low competence tend to overestimate their abilities. In our model, this
corresponds to \(K_0 = 0\) (ignorance) but \(K_1 = -1\) (misrecognition
of ignorance).

\textbf{Fleming and Daw (2017)} proposed a general Bayesian framework
for metacognitive computation, modeling metacognition as ``second-order
inference'' about the reliability of first-order cognitive processes.
Their distinction between first-order states and second-order inference
corresponds to our State\(_0\)/State\(_1\) hierarchy. While their
approach is Bayesian (modeling uncertainty about internal states) and
ours is observational (measuring alignment between claims and
performance), both frameworks capture the fundamental insight that
metacognition operates on a different level from cognition itself. The
K-C dissociation in our framework (epistemic state vs phenomenological
confidence) parallels their analysis of how confidence can diverge from
accuracy.

\textbf{Meta-d' (Maniscalco \& Lau, 2012)} provides a signal
detection-theoretic measure of metacognitive sensitivity. While meta-d'
quantifies \textbf{how well} subjects discriminate their own correct
from incorrect responses, our framework provides a \textbf{structural
vocabulary} for \textbf{what} metacognitive states exist. The two
approaches are complementary: meta-d' measures the quality of
monitoring; our \(K\) classifies the content of monitoring.

\textbf{Novel Contribution:} Our model provides a \textbf{structural
formalization} of recursive self-awareness that: 1. Unifies the
object/meta distinction (Nelson \& Narens) with a single recursive
operator 2. Separates epistemic state from cue-based confidence (Koriat)
3. Classifies all possible metacognitive configurations (27 patterns) 4.
Explicitly distinguishes ``Knowing Ignorance'' (Socratic wisdom) as a
high metacognitive achievement

\hypertarget{why-a-single-unified-k}{%
\subsubsection{Why a Single Unified K?}\label{why-a-single-unified-k}}

One might ask: ``Why not introduce separate operators for different
levels?'' The answer lies in the \textbf{universality of the epistemic
question}.

\textbf{Philosophical Motivation:}

The question ``Do I know?'' is the same question at every level: - ``Do
I know the answer?'' → \(K_0\) - ``Do I know whether I know the
answer?'' → \(K_1\) - ``Do I know whether I know whether I know?'' →
\(K_2\)

\textbf{The question is universal. Only the object changes.}

A thermometer measures temperature. The same thermometer can measure the
temperature of water, air, or metal. We do not need separate
``water-thermometers'' and ``air-thermometers.'' The instrument is the
same; the objects differ.

Similarly, \(K\) is an \textbf{observation protocol}, not a mental
process. The same protocol applies to different objects (State\(_0\),
State\(_1\), State\(_2\)). Introducing separate operators (\(R\), \(E\),
etc.) would obscure this fundamental unity and sacrifice the elegance of
a single recursive structure.

\textbf{Practical Implementation:}

While the \textbf{semantic anchors} are shared (-1/0/1 for
misconception/ignorance/knowledge), the \textbf{measurement procedures}
\(K^{(n)}\) may differ:

\begin{longtable}[]{@{}cll@{}}
\toprule\noalign{}
Layer & Observable & Measurement Procedure \\
\midrule\noalign{}
\endhead
\bottomrule\noalign{}
\endlastfoot
\(K^{(0)}\) & Response vs Reference & Accuracy scoring \\
\(K^{(1)}\) & Claim vs State\(_0\) & Alignment scoring \\
\(K^{(2)}\) & Meta-claim vs State\(_1\) & Meta-alignment scoring \\
\end{longtable}

\textbf{Parameter Tying (Optional):}

For parsimony, one may assume: - Same noise model across layers - Same
link function (e.g., logistic)

Or allow layer-specific parameters if data supports it.

\textbf{The key constraint is shared anchor semantics, not identical
functional forms.}

\textbf{The beauty of \(K\) is its universality.} Knowing, not knowing,
and misunderstanding are universal human experiences. The same operator
captures them all.

\hypertarget{measurement-theory}{%
\subsection{Measurement Theory}\label{measurement-theory}}

This section describes how the theoretical constructs (\(K(x)\),
\(K(K(x))\)) can be operationalized and measured empirically.

\hypertarget{why-continuous-scale}{%
\subsubsection{Why Continuous Scale?}\label{why-continuous-scale}}

The continuous scale \([-1, 1]\) provides several advantages over binary
or categorical representations:

\textbf{1. Intermediate States:}

Captures partial knowledge, uncertain beliefs, and mixed states that
binary representations cannot express.

\begin{itemize}
\tightlist
\item
  Example: \(K_0 = 0.3\) represents ``mostly ignorant but with some
  relevant information''
\item
  Example: \(K_1 = -0.5\) represents ``moderate overconfidence, not
  extreme''
\end{itemize}

\textbf{2. Change Tracking:}

Enables measurement of \textbf{gradual transitions and intervention
effects}.

\begin{itemize}
\tightlist
\item
  Example: After metacognitive training, \(K_1\) moves from \(-0.8\) to
  \(-0.2\)

  \begin{itemize}
  \tightlist
  \item
    This shows improvement within the ``overconfidence'' category
  \item
    Binary classification would show no change (both are
    ``overconfident'')
  \end{itemize}
\end{itemize}

\textbf{3. Aggregation:}

Permits meaningful averaging across items, domains, or time points.

\begin{itemize}
\tightlist
\item
  Example: Average \(K_1\) across 50 items yields a stable estimate
\item
  Example: Compare \(K_1\) across domains (math vs.~history)
\end{itemize}

\textbf{4. Statistical Modeling:}

Compatible with standard regression, Bayesian inference, and
psychometric methods.

\begin{itemize}
\tightlist
\item
  Linear models: \(K_1 \sim K_0 + \text{training} + \epsilon\)
\item
  Hierarchical models: Subject-level and item-level random effects
\end{itemize}

\textbf{5. Geometric Extension (Future Work):}

Enables connection to \textbf{information geometry} and manifold-based
analysis:

\begin{itemize}
\tightlist
\item
  Cognitive states as points on a manifold
\item
  Interventions as trajectories
\item
  Distance metrics for comparing metacognitive profiles
\item
  Curvature analysis for stability of states
\end{itemize}

\textbf{Design Choice:}

The trichotomy \(\{-1, 0, 1\}\) represents \textbf{prototypical anchors}
on the continuous scale, not the only valid values. Researchers may: -
Use discrete elicitation and embed into continuous scale - Use
probabilistic elicitation for direct continuous measurement - Aggregate
discrete responses to obtain continuous estimates

\hypertarget{measurement-theoretic-interpretation}{%
\subsubsection{Measurement-Theoretic
Interpretation}\label{measurement-theoretic-interpretation}}

Mathematically, all epistemic states live on a \textbf{single continuous
scale}:

\[K_n \in [-1, 1] \quad (n = 0, 1, 2, \dots)\]

The values \(-1\), \(0\), and \(1\) function as \textbf{prototypical
anchor points} on this continuum:

\begin{longtable}[]{@{}
  >{\centering\arraybackslash}p{(\columnwidth - 4\tabcolsep) * \real{0.3846}}
  >{\raggedright\arraybackslash}p{(\columnwidth - 4\tabcolsep) * \real{0.3077}}
  >{\raggedright\arraybackslash}p{(\columnwidth - 4\tabcolsep) * \real{0.3077}}@{}}
\toprule\noalign{}
\begin{minipage}[b]{\linewidth}\centering
Value
\end{minipage} & \begin{minipage}[b]{\linewidth}\raggedright
Prototype
\end{minipage} & \begin{minipage}[b]{\linewidth}\raggedright
Interpretation
\end{minipage} \\
\midrule\noalign{}
\endhead
\bottomrule\noalign{}
\endlastfoot
\(1\) & Full correct knowledge & Subject's state is maximally aligned
with the chosen reference \\
\(0\) & Pure ignorance & Subject has no determinate stance regarding the
object \\
\(-1\) & Full misconception & Subject's state is maximally opposed to
the reference \\
\end{longtable}

All intermediate values in \((-1, 0)\) and \((0, 1)\) represent
\textbf{graded mixtures} of these prototypes (partial knowledge, partial
misconception, uncertainty, mixtures across items, etc.).

\textbf{Operationalization Options (always mapping back to
\([-1, 1]\)):}

\begin{enumerate}
\def\labelenumi{\arabic{enumi}.}
\item
  \textbf{Discrete elicitation → Discrete embedding}: Use trichotomous
  responses (True/False/I don't know), then embed into \([-1, 1]\) via
  \(K(x) \in \{-1, 0, 1\}\) as prototype points.
\item
  \textbf{Probabilistic elicitation → Continuous embedding}: Elicit a
  subjective probability \(p(x)\) and map it into \([-1, 1]\) using a
  proper scoring rule or a simple linear transform (e.g., centered
  Brier-type scores).
\item
  \textbf{Aggregation → Continuous embedding}: Average prototype-valued
  \(K(x_i) \in \{-1, 0, 1\}\) across multiple items or contexts to
  obtain a continuous summary in \([-1, 1]\).
\end{enumerate}

Conceptually, the \textbf{continuum \([-1, 1]\) is primary}; the
trichotomy \(\{-1, 0, 1\}\) is a convenient way to name salient regions
on this line, not a separate codomain. Experimental designs may choose
discrete or continuous elicitation, but in all cases the resulting data
are interpreted as points (or distributions) on the same underlying
scale \([-1, 1]\).

\hypertarget{continuous-estimation-of-kkx}{%
\subsubsection{\texorpdfstring{Continuous Estimation of
\(K(K(x))\)}{Continuous Estimation of K(K(x))}}\label{continuous-estimation-of-kkx}}

The categorical inference of \(K(K(x))\) from a single ``Do you know?''
claim is a \textbf{simplified operationalization}. For more robust
measurement, we propose:

\textbf{Option 1: Aggregation Across Items}

For a subject responding to multiple items within a domain:

\[K(K)_{aggregate} = 2 \cdot P(\text{meta-claim matches actual state}) - 1\]

where \(P\) is estimated across all items. This yields a continuous
value in \([-1, 1]\).

\textbf{Option 2: Hierarchical Bayesian Estimation}

Model \(K(K(x))\) as a latent variable with: - Prior distribution over
subjects - Item-level random effects - Observation model linking latent
\(K(K(x))\) to categorical claims

This approach accommodates noise, individual differences, and item
difficulty.

\textbf{Option 3: Probabilistic Elicitation}

Instead of categorical ``Yes/No/Unsure'', elicit: - ``How confident are
you that your previous answer was correct?'' (0-100\%)

Map this to \(K(K(x))\) via a proper scoring rule or calibration
analysis.

\hypertarget{the-challenge-of-measuring-second-order-states}{%
\subsubsection{The Challenge of Measuring Second-Order
States}\label{the-challenge-of-measuring-second-order-states}}

\(K(K(x))\) is a \textbf{second-order epistemic state}: it represents
the subject's recognition of their own first-order state \(K(x)\). We
cannot directly observe \(K(K(x))\); we must infer it from observable
behavior.

\hypertarget{operational-definition-of-confidence-c}{%
\subsubsection{\texorpdfstring{Operational Definition of Confidence
(\(C\))}{Operational Definition of Confidence (C)}}\label{operational-definition-of-confidence-c}}

To fully characterize the phenomenological experience of metacognition,
we additionally measure \textbf{subjective confidence} \(C\).

\textbf{Definition:}

Confidence \(C\) is a \textbf{phenomenological self-report} of
subjective certainty, measured on a scale (e.g., 0-100\% or 1-7 Likert).

\[C(x) \in [0, 1] \quad \text{(or any bounded interval)}\]

\textbf{Key Distinction: K vs C}

\begin{longtable}[]{@{}lll@{}}
\toprule\noalign{}
Dimension & K (Epistemic State) & C (Confidence) \\
\midrule\noalign{}
\endhead
\bottomrule\noalign{}
\endlastfoot
\textbf{What it measures} & Alignment with reference & Subjective
feeling \\
\textbf{Anchor} & Correct/Incorrect/Absent & Certain/Uncertain \\
\textbf{Sign} & Signed (−1 to 1) & Unsigned (0 to 1) \\
\textbf{Basis} & External validation & Internal experience \\
\end{longtable}

\textbf{Orthogonality:}

\(K\) and \(C\) are \textbf{conceptually orthogonal}:

\begin{longtable}[]{@{}lccl@{}}
\toprule\noalign{}
Pattern & \(K_0\) & \(C\) & Interpretation \\
\midrule\noalign{}
\endhead
\bottomrule\noalign{}
\endlastfoot
Confident correct & 1 & High & Ideal \\
Confident wrong & −1 & High & Dangerous misconception \\
Unconfident correct & 1 & Low & Imposter-like \\
Unconfident wrong & −1 & Low & Appropriate uncertainty \\
\end{longtable}

\textbf{Diagnostic Role:}

\(C\) helps distinguish subtypes within the same \(K\) pattern: - DK
(\(K_0=0\), \(K_1=-1\)) with \textbf{high \(C\)} → Overconfident
ignorance - DK (\(K_0=0\), \(K_1=-1\)) with \textbf{low \(C\)} →
Uncertain but still wrong claim

\textbf{K-C Dissociation Hypothesis:}

Subjects can have: - High \(K_0\) with low \(C\) (Imposter syndrome) -
Low \(K_0\) with high \(C\) (Dunning-Kruger effect)

This dissociation is empirically testable and clinically meaningful.

\textbf{Measurement Protocol for C:}

\begin{enumerate}
\def\labelenumi{\arabic{enumi}.}
\tightlist
\item
  Elicit response → compute \(K_0\)
\item
  Elicit confidence (0-100\%) → record \(C\)
\item
  Elicit metacognitive claim (``Do you know?'') → compute \(K_1\)
\item
  Analyze \(K \times C\) jointly for full characterization
\end{enumerate}

\textbf{Important Distinction (Summary):} - \(K(x)\): Epistemic state
(how accurately the subject recognizes \(x\)) - \(C\): Phenomenological
confidence (how certain the subject feels)

These are \textbf{orthogonal dimensions}. A subject can have: -
\(K(x) = 0\) (ignorance) with \(C = 1\) (high confidence) ---
Dunning-Kruger - \(K(x) = 1\) (knowledge) with \(C = 0.5\) (moderate
confidence) --- Underconfidence

\hypertarget{measurement-protocol}{%
\subsubsection{Measurement Protocol}\label{measurement-protocol}}

\hypertarget{step-1-establish-reference-context}{%
\paragraph{Step 1: Establish Reference
Context}\label{step-1-establish-reference-context}}

For each proposition \(x\), establish what counts as ``aligned''
(\(K(x) = 1\)) via the experimental context (e.g., expert consensus,
empirical measurement, community agreement). The proposition \(x\)
itself serves as the implicit reference point.

\hypertarget{step-2-measure-kx-via-task-performance}{%
\paragraph{\texorpdfstring{Step 2: Measure \(K(x)\) via Task
Performance}{Step 2: Measure K(x) via Task Performance}}\label{step-2-measure-kx-via-task-performance}}

\textbf{Task:} Subject answers: ``Is proposition \(x\) true, false, or
unknown?''

\begin{longtable}[]{@{}lcc@{}}
\toprule\noalign{}
Subject's Answer & Reference & Inferred \(K(x)\) \\
\midrule\noalign{}
\endhead
\bottomrule\noalign{}
\endlastfoot
``True'' & Aligned & \(1\) (correct knowledge) \\
``False'' & Aligned (proposition is false) & \(1\) (correct
knowledge) \\
``I don't know'' & any & \(0\) (ignorance) \\
``True'' & Opposed & \(-1\) (misconception) \\
``False'' & Opposed (proposition is true) & \(-1\) (misconception) \\
\end{longtable}

\hypertarget{step-3-measure-confidence-c_0}{%
\paragraph{\texorpdfstring{Step 3: Measure Confidence
\(C_0\)}{Step 3: Measure Confidence C\_0}}\label{step-3-measure-confidence-c_0}}

\textbf{Question:} ``On a scale from 0 to 1, how confident are you in
your answer?''

This captures the phenomenological dimension of certainty.

\hypertarget{step-4-elicit-metacognitive-claim}{%
\paragraph{Step 4: Elicit Metacognitive
Claim}\label{step-4-elicit-metacognitive-claim}}

\textbf{Question:} ``Do you know the answer to the previous question?''

\begin{longtable}[]{@{}ll@{}}
\toprule\noalign{}
Subject's Claim & Interpretation \\
\midrule\noalign{}
\endhead
\bottomrule\noalign{}
\endlastfoot
``Yes, I know'' & Subject claims \(K(K(x)) = 1\) \\
``No, I don't know'' & Subject claims \(K(K(x)) = 0\) \\
``I'm not sure'' & Subject claims \(K(K(x)) \approx 0.5\) \\
\end{longtable}

\hypertarget{step-5-infer-actual-kkx-via-comparison}{%
\paragraph{\texorpdfstring{Step 5: Infer Actual \(K(K(x))\) via
Comparison}{Step 5: Infer Actual K(K(x)) via Comparison}}\label{step-5-infer-actual-kkx-via-comparison}}

Compare the subject's \textbf{metacognitive claim} (Step 4) to their
\textbf{actual state} (Step 2):

\begin{longtable}[]{@{}
  >{\centering\arraybackslash}p{(\columnwidth - 6\tabcolsep) * \real{0.2778}}
  >{\raggedright\arraybackslash}p{(\columnwidth - 6\tabcolsep) * \real{0.2222}}
  >{\centering\arraybackslash}p{(\columnwidth - 6\tabcolsep) * \real{0.2778}}
  >{\raggedright\arraybackslash}p{(\columnwidth - 6\tabcolsep) * \real{0.2222}}@{}}
\toprule\noalign{}
\begin{minipage}[b]{\linewidth}\centering
Actual \(K(x)\)
\end{minipage} & \begin{minipage}[b]{\linewidth}\raggedright
Subject's Claim
\end{minipage} & \begin{minipage}[b]{\linewidth}\centering
Inferred \(K(K(x))\)
\end{minipage} & \begin{minipage}[b]{\linewidth}\raggedright
Classification
\end{minipage} \\
\midrule\noalign{}
\endhead
\bottomrule\noalign{}
\endlastfoot
\(1\) (knows) & ``I know'' & \(1\) & \textbf{Knowing Knowledge} \\
\(0\) (ignorant) & ``I don't know'' & \(1\) & \textbf{Knowing Ignorance}
(Socratic) \\
\(0\) (ignorant) & ``I know'' & \(-1\) & \textbf{Unknowing Ignorance}
(Dunning-Kruger) \\
\(1\) (knows) & ``I don't know'' & \(-1\) or \(0\) & \textbf{Unknowing
Knowledge} (Imposter) \\
\end{longtable}

\textbf{Key Insight:} \(K(K(x))\) is inferred by checking whether the
subject's \textbf{metacognitive claim matches their actual state}.

\hypertarget{analyzing-discrepancies}{%
\subsubsection{Analyzing Discrepancies}\label{analyzing-discrepancies}}

\hypertarget{metacognitive-discrepancy}{%
\paragraph{Metacognitive Discrepancy}\label{metacognitive-discrepancy}}

The discrepancy between actual state and metacognitive claim is captured
directly by \(K(K(x))\): - \(K(K(x)) = 1\): Accurate metacognition
(claim matches reality) - \(K(K(x)) = 0\): Partial metacognitive failure
- \(K(K(x)) = -1\): Complete metacognitive failure (claim contradicts
reality)

\hypertarget{experimental-design-the-metacognitive-alignment-test-mat}{%
\subsection{Experimental Design: The Metacognitive Alignment Test
(MAT)}\label{experimental-design-the-metacognitive-alignment-test-mat}}

To demonstrate the falsifiability and measurability of this model, we
propose the \textbf{Metacognitive Alignment Test (MAT)}.

\hypertarget{objectives}{%
\subsubsection{Objectives}\label{objectives}}

\begin{enumerate}
\def\labelenumi{\arabic{enumi}.}
\tightlist
\item
  Measure \(K(x)\) (first-order epistemic state)
\item
  Measure \(K(K(x))\) (second-order metacognitive state)
\item
  Measure confidence \(C\) (phenomenological dimension)
\item
  Validate the distinction between Socratic Wisdom and Dunning-Kruger
  effect
\end{enumerate}

\hypertarget{protocol}{%
\subsubsection{Protocol}\label{protocol}}

\textbf{Phase 1: Knowledge Assessment} - Present factual questions with
established reference answers (e.g., expert consensus) - Subject
responds: True / False / I don't know - Calculate \(K(x)\) based on
alignment with reference

\textbf{Phase 2: Confidence Rating} - Subject rates confidence: ``How
confident are you?'' (0-1 scale) - Record \(C_0\)

\textbf{Phase 3: Metacognitive Claim} - Ask: ``Do you know the answer to
the previous question?'' - Subject responds: Yes / No / Unsure - Infer
\(K(K(x))\) by comparing claim to actual \(K(x)\)

\textbf{Phase 4: Validation Tasks} - Present decision-making scenarios
requiring self-assessment - Measure performance on tasks like: -
Deciding when to seek help - Allocating study time - Deferring to
experts

\hypertarget{validation-hypothesis}{%
\subsubsection{Validation Hypothesis}\label{validation-hypothesis}}

\textbf{Hypothesis:} Subjects with high \(K(K(x))\) (accurate
metacognition) will perform better on validation tasks,
\textbf{regardless of their raw \(K(x)\) score}.

This would validate the model's claim that: - \textbf{Knowing Ignorance}
(\(K(x) = 0, K(K(x)) = 1\)) is a valuable cognitive state -
Metacognitive accuracy is distinct from first-order knowledge - Socratic
wisdom has measurable benefits

\hypertarget{expected-patterns}{%
\subsubsection{Expected Patterns}\label{expected-patterns}}

\begin{longtable}[]{@{}
  >{\raggedright\arraybackslash}p{(\columnwidth - 8\tabcolsep) * \real{0.1739}}
  >{\centering\arraybackslash}p{(\columnwidth - 8\tabcolsep) * \real{0.2174}}
  >{\centering\arraybackslash}p{(\columnwidth - 8\tabcolsep) * \real{0.2174}}
  >{\centering\arraybackslash}p{(\columnwidth - 8\tabcolsep) * \real{0.2174}}
  >{\raggedright\arraybackslash}p{(\columnwidth - 8\tabcolsep) * \real{0.1739}}@{}}
\toprule\noalign{}
\begin{minipage}[b]{\linewidth}\raggedright
Pattern
\end{minipage} & \begin{minipage}[b]{\linewidth}\centering
\(K(x)\)
\end{minipage} & \begin{minipage}[b]{\linewidth}\centering
\(K(K(x))\)
\end{minipage} & \begin{minipage}[b]{\linewidth}\centering
\(C\)
\end{minipage} & \begin{minipage}[b]{\linewidth}\raggedright
Expected Behavior
\end{minipage} \\
\midrule\noalign{}
\endhead
\bottomrule\noalign{}
\endlastfoot
Socratic Wisdom & \(0\) & \(1\) & Low & Seeks help appropriately \\
Dunning-Kruger & \(0\) & \(-1\) & High & Overconfident errors \\
Accurate Expert & \(1\) & \(1\) & High & Confident and correct \\
Imposter Syndrome & \(1\) & \(-1\) or \(0\) & Low & Underconfident but
correct \\
\end{longtable}

\hypertarget{relationship-to-established-metrics}{%
\subsubsection{Relationship to Established
Metrics}\label{relationship-to-established-metrics}}

The MAT is designed to \textbf{complement, not replace}, existing
metacognitive measures:

\begin{longtable}[]{@{}
  >{\raggedright\arraybackslash}p{(\columnwidth - 4\tabcolsep) * \real{0.3333}}
  >{\raggedright\arraybackslash}p{(\columnwidth - 4\tabcolsep) * \real{0.3333}}
  >{\raggedright\arraybackslash}p{(\columnwidth - 4\tabcolsep) * \real{0.3333}}@{}}
\toprule\noalign{}
\begin{minipage}[b]{\linewidth}\raggedright
Metric
\end{minipage} & \begin{minipage}[b]{\linewidth}\raggedright
What it measures
\end{minipage} & \begin{minipage}[b]{\linewidth}\raggedright
Relationship to MAT
\end{minipage} \\
\midrule\noalign{}
\endhead
\bottomrule\noalign{}
\endlastfoot
\textbf{meta-d'} & Metacognitive sensitivity (discrimination) & Can be
computed from MAT data; provides aggregate validation \\
\textbf{Brier Score} & Probabilistic calibration & Applicable if
confidence is elicited as probability \\
\textbf{ECE} & Expected calibration error & Measures bias in
confidence-accuracy relationship \\
\textbf{AUROC} & Discrimination ability & Can be derived from confidence
ratings vs.~accuracy \\
\textbf{IRT} & Item difficulty and discrimination & Can model item-level
variance in MAT responses \\
\end{longtable}

\textbf{Recommended Analysis Pipeline:}

\begin{enumerate}
\def\labelenumi{\arabic{enumi}.}
\tightlist
\item
  Compute \(K(x)\) and \(K(K(x))\) per item using MAT protocol
\item
  Compute meta-d' across trials as aggregate metacognitive sensitivity
\item
  Compute calibration metrics (Brier, ECE) from confidence ratings
\item
  Compare \(K(K(x))\) patterns (Socratic, Dunning-Kruger, etc.) with
  meta-d' to validate convergent validity
\item
  Use IRT to account for item-level heterogeneity
\end{enumerate}

\textbf{Hypothesis:} High \(K(K(x))\) (accurate metacognition) should
correlate with high meta-d'/d' ratio and good calibration, but
\(K(K(x))\) provides additional structural information (e.g.,
distinguishing Socratic wisdom from mere low confidence).

\hypertarget{related-work}{%
\subsection{Related Work}\label{related-work}}

\hypertarget{relationship-to-epistemic-logic}{%
\subsubsection{Relationship to Epistemic
Logic}\label{relationship-to-epistemic-logic}}

Traditional epistemic logics (e.g., S5, KD45) model knowledge via modal
operators with introspection axioms:

\begin{itemize}
\tightlist
\item
  \textbf{Positive Introspection}: \(Kp \to KKp\) (``If I know \(p\), I
  know that I know \(p\)'')
\item
  \textbf{Negative Introspection}: \(\neg Kp \to K\neg Kp\) (``If I
  don't know \(p\), I know that I don't know \(p\)'')
\end{itemize}

\textbf{Comparison to Our Framework:}

\begin{longtable}[]{@{}
  >{\raggedright\arraybackslash}p{(\columnwidth - 4\tabcolsep) * \real{0.3333}}
  >{\raggedright\arraybackslash}p{(\columnwidth - 4\tabcolsep) * \real{0.3333}}
  >{\raggedright\arraybackslash}p{(\columnwidth - 4\tabcolsep) * \real{0.3333}}@{}}
\toprule\noalign{}
\begin{minipage}[b]{\linewidth}\raggedright
Aspect
\end{minipage} & \begin{minipage}[b]{\linewidth}\raggedright
Epistemic Logic
\end{minipage} & \begin{minipage}[b]{\linewidth}\raggedright
Our \(K(K(x))\)
\end{minipage} \\
\midrule\noalign{}
\endhead
\bottomrule\noalign{}
\endlastfoot
\textbf{Representation} & Binary (knows/doesn't know) & Continuous
\([-1, 1]\) \\
\textbf{Misconception} & Not modeled & \(K(x) = -1\) \\
\textbf{Metacognitive failure} & Violates introspection axioms &
\(K(K(x)) \neq \text{sign}(K(x))\) \\
\textbf{Dunning-Kruger} & Not expressible &
\(K(x) = 0, K(K(x)) = -1\) \\
\end{longtable}

\textbf{Key Departure:} Epistemic logics typically assume idealized
agents with perfect introspection. Our framework explicitly models
\textbf{failures of introspection}---cases where
\(K(K(x)) \neq \text{sign}(K(x))\). This captures the Dunning-Kruger
effect and imposter syndrome, which are empirically observed but cannot
be expressed in standard epistemic logics without violating the
introspection axioms.

Our approach can be seen as a \textbf{graded, psychologically realistic}
extension of epistemic logic that relaxes the introspection axioms to
accommodate metacognitive failures.

\hypertarget{relationship-to-graded-epistemic-logics}{%
\subsubsection{Relationship to Graded Epistemic
Logics}\label{relationship-to-graded-epistemic-logics}}

Recent work in graded epistemic logics (e.g., S5G frameworks) models
knowledge with continuous plausibility values in \([0, 1]\). Our
framework differs in key respects:

\begin{longtable}[]{@{}
  >{\raggedright\arraybackslash}p{(\columnwidth - 4\tabcolsep) * \real{0.3333}}
  >{\raggedright\arraybackslash}p{(\columnwidth - 4\tabcolsep) * \real{0.3333}}
  >{\raggedright\arraybackslash}p{(\columnwidth - 4\tabcolsep) * \real{0.3333}}@{}}
\toprule\noalign{}
\begin{minipage}[b]{\linewidth}\raggedright
Aspect
\end{minipage} & \begin{minipage}[b]{\linewidth}\raggedright
Graded Epistemic Logics
\end{minipage} & \begin{minipage}[b]{\linewidth}\raggedright
Our \(K(K(x))\)
\end{minipage} \\
\midrule\noalign{}
\endhead
\bottomrule\noalign{}
\endlastfoot
\textbf{Scale} & \([0, 1]\) (plausibility) & \([-1, 1]\) (includes
misconception) \\
\textbf{Misconception} & Typically not modeled & \(K(x) = -1\) \\
\textbf{Metacognition} & Introspection axioms & Explicit recursive
operator \\
\textbf{Focus} & Idealized agents & Psychologically realistic
failures \\
\end{longtable}

\textbf{Formal Correspondence:}

Our \(K\) can be viewed as a \textbf{graded, psychologically realistic}
extension that:

\begin{enumerate}
\def\labelenumi{\arabic{enumi}.}
\tightlist
\item
  Relaxes positive/negative introspection axioms to accommodate
  metacognitive failures
\item
  Extends the scale to include misconception (negative values)
\item
  Focuses on the \textbf{gap} between actual and recognized epistemic
  states, rather than idealized consistency
\end{enumerate}

Whether \(K\) is idempotent (\(K(K(x)) = K(x)\) for accurate
metacognizers) or contractive (higher-order reflection converges) is an
\textbf{empirical question} that our framework can accommodate but does
not presuppose.

\hypertarget{metacognitive-sensitivity-meta-d}{%
\subsubsection{Metacognitive Sensitivity:
meta-d'}\label{metacognitive-sensitivity-meta-d}}

Maniscalco and Lau (2012) developed the \emph{meta-d'} framework for
measuring metacognitive sensitivity---the ability to discriminate
between correct and incorrect responses via confidence ratings.

\textbf{Comparison:}

\begin{longtable}[]{@{}
  >{\raggedright\arraybackslash}p{(\columnwidth - 4\tabcolsep) * \real{0.3333}}
  >{\raggedright\arraybackslash}p{(\columnwidth - 4\tabcolsep) * \real{0.3333}}
  >{\raggedright\arraybackslash}p{(\columnwidth - 4\tabcolsep) * \real{0.3333}}@{}}
\toprule\noalign{}
\begin{minipage}[b]{\linewidth}\raggedright
Aspect
\end{minipage} & \begin{minipage}[b]{\linewidth}\raggedright
meta-d'
\end{minipage} & \begin{minipage}[b]{\linewidth}\raggedright
Our \(K(K(x))\)
\end{minipage} \\
\midrule\noalign{}
\endhead
\bottomrule\noalign{}
\endlastfoot
\textbf{Focus} & Discrimination ability (sensitivity) & Structural
accuracy (recognition) \\
\textbf{Measurement} & Statistical correlation across trials & Per-item
metacognitive state \\
\textbf{Granularity} & Aggregate across trials & Per-item \\
\textbf{``I don't know''} & Treated as low confidence &
\(K(x)=0, K(K(x))=1\) (Socratic wisdom) \\
\textbf{Statistical Model} & Noise-tolerant (SDT) & Deterministic
match/mismatch \\
\textbf{Theoretical Basis} & Signal Detection Theory & Recursive
epistemology \\
\end{longtable}

\textbf{Key Difference:} meta-d' measures whether confidence ratings
\textbf{correlate} with accuracy. Our model measures whether
metacognitive claims \textbf{match} actual states. Crucially, we
recognize that \textbf{accurately knowing one's ignorance}
(\(K(x) = 0, K(K(x)) = 1\)) is a \textbf{high metacognitive
achievement}, not a failure.

\textbf{Complementary Relationship:} These approaches are \textbf{not
mutually exclusive}. meta-d' provides a noise-tolerant aggregate measure
of metacognitive sensitivity; our \(K(K(x))\) provides per-item
structural classification with explicit treatment of Socratic wisdom. An
integrated approach could: - Use meta-d' for aggregate sensitivity
analysis across trials - Use \(K(K(x))\) for per-item classification and
Socratic wisdom detection - Define a continuous version:
\(K(K(x)) = 2 \cdot P(\text{meta-claim matches actual state}) - 1\),
estimated across trials via hierarchical Bayesian methods

\hypertarget{information-theoretic-metacognition-meta-i}{%
\subsubsection{Information-Theoretic Metacognition:
meta-I}\label{information-theoretic-metacognition-meta-i}}

Dayan (2023) introduced \emph{meta-I}, a \textbf{model-free}
information-theoretic measure of metacognitive sensitivity:

\[\text{meta-I} = H(\text{accuracy}) - H(\text{accuracy} | \text{confidence})\]

This measures mutual information between confidence and accuracy,
quantifying how much confidence reduces uncertainty about accuracy.

\textbf{Comparison:}

\begin{longtable}[]{@{}
  >{\raggedright\arraybackslash}p{(\columnwidth - 4\tabcolsep) * \real{0.3333}}
  >{\raggedright\arraybackslash}p{(\columnwidth - 4\tabcolsep) * \real{0.3333}}
  >{\raggedright\arraybackslash}p{(\columnwidth - 4\tabcolsep) * \real{0.3333}}@{}}
\toprule\noalign{}
\begin{minipage}[b]{\linewidth}\raggedright
Aspect
\end{minipage} & \begin{minipage}[b]{\linewidth}\raggedright
meta-I
\end{minipage} & \begin{minipage}[b]{\linewidth}\raggedright
Our \(K\) Framework
\end{minipage} \\
\midrule\noalign{}
\endhead
\bottomrule\noalign{}
\endlastfoot
\textbf{Theoretical Basis} & Information Theory & Recursive
Epistemology \\
\textbf{Model Dependency} & Model-free & Model-free (observational) \\
\textbf{Direction} & Unsigned (sensitivity only) & \textbf{Signed}
(over/under-confidence) \\
\textbf{Layers} & Single layer & \textbf{Recursive}
(\(K_0, K_1, K_2, \ldots\)) \\
\textbf{Output} & Bits (continuous) & \([-1, 1]\) (continuous) \\
\textbf{Granularity} & Can measure response granularity & Per-item
structural classification \\
\end{longtable}

\textbf{Key Distinction:}

Both meta-I and \(K\) are \textbf{model-free} (unlike meta-d' which
requires SDT assumptions), but they serve different purposes:

\begin{itemize}
\tightlist
\item
  \textbf{meta-I} answers: ``How well does confidence track accuracy?''
  (quantitative sensitivity)
\item
  \textbf{\(K\)} answers: ``What type of metacognitive pattern is
  this?'' (qualitative classification)
\end{itemize}

\textbf{Complementary Use Case:}

Two subjects with identical meta-I = 0.3 bits (low sensitivity): -
Subject A: \(K_0=0\), \(K_1=-1\) → Dunning-Kruger → needs awareness
intervention - Subject B: \(K_0=1\), \(K_1=-1\) → Imposter → needs
confidence-building

meta-I cannot distinguish these cases; \(K\) can.

\textbf{Analogy:} - meta-d' / meta-I = \textbf{Thermometer} (measures
metacognitive temperature) - \(K\) = \textbf{Weather map} (classifies
metacognitive patterns, guides intervention)

\hypertarget{calibration-metrics-brier-score-ece}{%
\subsubsection{Calibration Metrics (Brier Score,
ECE)}\label{calibration-metrics-brier-score-ece}}

Calibration metrics measure whether confidence aligns with accuracy
across many trials.

\textbf{Comparison:}

\begin{longtable}[]{@{}lcl@{}}
\toprule\noalign{}
Aspect & Calibration Metrics & Our \(K(K(x))\) \\
\midrule\noalign{}
\endhead
\bottomrule\noalign{}
\endlastfoot
\textbf{Granularity} & Aggregate statistics & Individual items \\
\textbf{Purpose} & Probabilistic accuracy & Epistemic state
recognition \\
\textbf{Socratic Wisdom} & Not explicitly modeled & Explicitly
formalized \\
\end{longtable}

\textbf{Complementary Relationship:} Calibration metrics and \(K(K(x))\)
measure different aspects of metacognition. A subject can have good
calibration (confidence matches accuracy on average) but poor
\(K(K(x))\) on specific items (e.g., confidently wrong about specific
facts).

\hypertarget{belief-functions-and-uncertainty-dempster-shafer-theory}{%
\subsubsection{Belief Functions and Uncertainty (Dempster-Shafer
Theory)}\label{belief-functions-and-uncertainty-dempster-shafer-theory}}

Dempster-Shafer theory handles \textbf{epistemic uncertainty} and
\textbf{conflicting evidence} via belief functions.

\textbf{Comparison:}

\begin{longtable}[]{@{}
  >{\raggedright\arraybackslash}p{(\columnwidth - 4\tabcolsep) * \real{0.3333}}
  >{\raggedright\arraybackslash}p{(\columnwidth - 4\tabcolsep) * \real{0.3333}}
  >{\raggedright\arraybackslash}p{(\columnwidth - 4\tabcolsep) * \real{0.3333}}@{}}
\toprule\noalign{}
\begin{minipage}[b]{\linewidth}\raggedright
Aspect
\end{minipage} & \begin{minipage}[b]{\linewidth}\raggedright
Dempster-Shafer
\end{minipage} & \begin{minipage}[b]{\linewidth}\raggedright
Our Model
\end{minipage} \\
\midrule\noalign{}
\endhead
\bottomrule\noalign{}
\endlastfoot
\textbf{Focus} & Uncertainty quantification & Metacognitive
discrepancy \\
\textbf{Application} & Evidence combination & Self-awareness
structure \\
\textbf{Ignorance} & Represented as belief mass & \(K(x) = 0\)
(epistemic state) \\
\end{longtable}

\textbf{Complementary Relationship:} Dempster-Shafer theory could be
used to model uncertainty about the reference point when expert
consensus is incomplete. Our model specifically targets the \textbf{gap
between what one knows and what one thinks one knows}.

\hypertarget{dunning-kruger-effect-empirical-psychology}{%
\subsubsection{Dunning-Kruger Effect (Empirical
Psychology)}\label{dunning-kruger-effect-empirical-psychology}}

Kruger and Dunning (1999) empirically demonstrated that low-competence
individuals overestimate their abilities.

\textbf{Our Contribution:} We provide a \textbf{formal mathematical
model} of this phenomenon: - \(K(x) = 0\) (low competence) -
\(K(K(x)) = -1\) (misrecognition: believes they have competence) - Often
accompanied by \(C = 1\) (high confidence)

This formalization enables: 1. Precise measurement protocols 2.
Distinction from related phenomena (e.g., imposter syndrome) 3.
Extension to arbitrary depths of self-reflection

\hypertarget{application-to-ai-metacognition}{%
\subsubsection{Application to AI
Metacognition}\label{application-to-ai-metacognition}}

The \(K\) framework provides a structured vocabulary for evaluating
metacognition in Large Language Models (LLMs), an increasingly important
area as AI systems are deployed in high-stakes domains.

\textbf{Mapping LLM Behaviors:}

\begin{longtable}[]{@{}
  >{\raggedright\arraybackslash}p{(\columnwidth - 8\tabcolsep) * \real{0.1739}}
  >{\centering\arraybackslash}p{(\columnwidth - 8\tabcolsep) * \real{0.2174}}
  >{\centering\arraybackslash}p{(\columnwidth - 8\tabcolsep) * \real{0.2174}}
  >{\centering\arraybackslash}p{(\columnwidth - 8\tabcolsep) * \real{0.2174}}
  >{\raggedright\arraybackslash}p{(\columnwidth - 8\tabcolsep) * \real{0.1739}}@{}}
\toprule\noalign{}
\begin{minipage}[b]{\linewidth}\raggedright
LLM Pattern
\end{minipage} & \begin{minipage}[b]{\linewidth}\centering
\(K_0\)
\end{minipage} & \begin{minipage}[b]{\linewidth}\centering
\(K_1\)
\end{minipage} & \begin{minipage}[b]{\linewidth}\centering
\(K_2\)
\end{minipage} & \begin{minipage}[b]{\linewidth}\raggedright
Interpretation
\end{minipage} \\
\midrule\noalign{}
\endhead
\bottomrule\noalign{}
\endlastfoot
Correct + confident & 1 & 1 & 1 & Ideal calibration \\
Hallucination + confident & −1 & −1 & ? & Confident wrong (dangerous) \\
Correct + hedging & 1 & −1 & ? & Underconfident (imposter-like) \\
Admits uncertainty & 0 & 1 & 1 & Appropriate uncertainty (Socratic) \\
``I don't know'' when wrong & −1 & 1 & ? & Partial awareness of
limits \\
\end{longtable}

\textbf{Testbed Proposal:}

Using decoupled confidence elicitation (analogous to AFCE-style
protocols):

\begin{enumerate}
\def\labelenumi{\arabic{enumi}.}
\tightlist
\item
  \textbf{Query LLM for answer} → compute \(K_0\) (against ground truth)
\item
  \textbf{Query LLM for confidence} → record \(C\) (0-100\%)
\item
  \textbf{Query LLM: ``Do you know this?''} → elicit \(\text{Claim}_1\)
\item
  \textbf{Compute \(K_1\)} from \(\text{Claim}_1\) vs \(K_0\)
\end{enumerate}

This protocol allows testing: - \textbf{K vs C dissociation} in LLMs (do
they exhibit Dunning-Kruger or Imposter patterns?) -
\textbf{Domain-specific calibration} (are LLMs more self-aware in some
domains?) - \textbf{Intervention effects} (does prompting for
self-reflection improve \(K_1\)?)

\textbf{Why K is Useful for LLM Evaluation:}

Current LLM calibration research focuses primarily on
confidence-accuracy correlation (analogous to meta-d' or meta-I). The
\(K\) framework adds:

\begin{enumerate}
\def\labelenumi{\arabic{enumi}.}
\tightlist
\item
  \textbf{Pattern classification}: Identifying \emph{which type} of
  miscalibration
\item
  \textbf{Directional information}: Distinguishing overconfidence from
  underconfidence
\item
  \textbf{Intervention guidance}: Suggesting targeted prompting
  strategies
\end{enumerate}

\textbf{Future Work:} Operationalizing \(K_n\) for LLMs using abstention
behavior, self-consistency checks, and metamorphic testing remains an
open challenge (see Limitations).

\hypertarget{novel-contributions}{%
\subsubsection{Novel Contributions}\label{novel-contributions}}

This study is novel in:

\begin{enumerate}
\def\labelenumi{\arabic{enumi}.}
\tightlist
\item
  \textbf{Recursive Formalization}: Extending metacognition to arbitrary
  depths (\(K \to K(K) \to K(K(K))\))
\item
  \textbf{Socratic Wisdom as Achievement}: Explicitly modeling ``knowing
  ignorance'' as \(K(x)=0, K(K(x))=1\)
\item
  \textbf{Type-Theoretic Foundation}: Justifying \(K(K(x))\) via
  recursive types
\item
  \textbf{Orthogonal Dimensions}: Separating epistemic state (\(K\))
  from phenomenological confidence (\(C\))
\item
  \textbf{Per-Item Granularity}: Measuring metacognition at the
  individual item level, not just aggregate statistics
\end{enumerate}

\hypertarget{conclusion-and-future-challenges}{%
\subsection{Conclusion and Future
Challenges}\label{conclusion-and-future-challenges}}

This study constructed a recursive epistemic model based on the
hierarchical structure of knowledge. Using a single core function \(K\)
applied recursively---\(K(x)\), \(K(K(x))\), \(K(K(K(x)))\)---we provide
a mathematically rigorous yet philosophically grounded framework that
captures:

\begin{enumerate}
\def\labelenumi{\arabic{enumi}.}
\tightlist
\item
  The \textbf{recursive nature of self-awareness}: The same epistemic
  question applies at every level of reflection.
\item
  The \textbf{hierarchical structure of metacognition}: Distinguishing
  ``knowing ignorance'' (Socratic wisdom) from ``unknowing ignorance''
  (Dunning-Kruger effect).
\item
  The \textbf{continuous gradation of knowledge}: Knowledge states exist
  on a continuum from misconception (\(-1\)) through ignorance (\(0\))
  to accurate knowledge (\(1\)).
\end{enumerate}

\hypertarget{main-results}{%
\subsubsection{Main Results}\label{main-results}}

\begin{enumerate}
\def\labelenumi{\arabic{enumi}.}
\tightlist
\item
  We proposed a \textbf{recursive epistemic function \(K\)} with
  type-theoretic justification and axiomatic constraints (anchor
  preservation, monotonicity, boundedness), demonstrating that
  \(K(K(x))\) is not a type error but a well-founded recursive
  structure.
\item
  We adopted a stance of \textbf{methodological relativism}, treating
  the proposition \(x\) itself as the implicit reference point and
  leaving the designation of ``correct'' to the experimental context.
\item
  We provided the \textbf{Four Quadrants of Metacognition}, clearly
  distinguishing ``Knowing Ignorance'' (Socratic wisdom,
  \(K(x)=0, K(K(x))=1\)) from ``Unknowing Ignorance'' (Dunning-Kruger
  effect, \(K(x)=0, K(K(x))=-1\)).
\item
  We separated \textbf{epistemic state} (\(K\)) from
  \textbf{phenomenological confidence} (\(C\)), recognizing them as
  orthogonal dimensions.
\item
  We proposed the \textbf{Metacognitive Alignment Test (MAT)} as an
  experimental protocol to validate the model, with specific predictions
  about the benefits of Socratic wisdom.
\end{enumerate}

\hypertarget{theoretical-contributions}{%
\subsubsection{Theoretical
Contributions}\label{theoretical-contributions}}

\begin{itemize}
\tightlist
\item
  \textbf{Recursive Formalization}: Extending metacognition to arbitrary
  depths while maintaining mathematical consistency
\item
  \textbf{Type-Theoretic Foundation}: Justifying self-application via
  recursive types
\item
  \textbf{Socratic Wisdom as Achievement}: Explicitly modeling ``knowing
  ignorance'' as a high metacognitive state
\item
  \textbf{Per-Item Granularity}: Measuring metacognition at the
  individual item level, complementing aggregate statistical measures
\end{itemize}

\hypertarget{limitations}{%
\subsubsection{Limitations}\label{limitations}}

This framework is a \textbf{conceptual scaffold} for organizing
metacognitive phenomena, not a complete predictive model. We acknowledge
the following limitations:

\begin{enumerate}
\def\labelenumi{\arabic{enumi}.}
\item
  \textbf{Formal Model:} We adopt an observational family interpretation
  to resolve the recursion/observation tension. This is a modeling
  choice, not the only possibility.
\item
  \textbf{Single Dimension (Scope Boundary):}

  \begin{itemize}
  \tightlist
  \item
    The \([-1,1]\) scale assumes a \textbf{single axis of correctness}
    with one ``opposite'' direction.
  \item
    \textbf{What this captures:} Directional errors (overconfidence vs
    underconfidence, correct vs incorrect).
  \item
    \textbf{What this does NOT capture:} Multiple, qualitatively
    different misconceptions (e.g., ``thinks A'' vs ``thinks B'' when
    truth is C).
  \item
    \textbf{Design rationale:} This simplification enables tractable
    measurement and clear intervention design. Multi-dimensional
    misconceptions require a different formalism (e.g., vector-valued
    \(K\) or belief distributions) and are outside the current scope.
  \item
    \textbf{Future extension:} Vector-valued \(K \in [-1,1]^d\) or
    embedding in a latent space could address this limitation.
  \end{itemize}
\item
  \textbf{Minimal Axiomatic Theory:} We provide basic constraints on
  \(K\) (anchor preservation, monotonicity, boundedness) but do not
  specify a unique functional form. The specific dynamics of \(K\)
  (e.g., whether it is contractive, has fixed points beyond
  \(\{-1, 0, 1\}\)) are empirical questions.
\item
  \textbf{No Generative Model:} We do not provide a noise model or
  generative account of how states are produced. This is a task for
  computational cognitive modeling.
\item
  \textbf{Observation Protocol:} The mapping from observable behavior to
  \(K_n\) values requires operational definitions. While we provide
  guidelines (e.g., alignment between claims and performance), the
  specific elicitation methods depend on the application domain.
\item
  \textbf{\(K_2\) Identifiability:} Higher-order estimates (\(K_2\),
  \(K_3\)) require more items and may be less reliable.

  \begin{itemize}
  \tightlist
  \item
    \textbf{Guideline:} For reliable \(K_2\) estimation, use
    \(N \geq 50\) items with noise \(< 0.2\).
  \item
    \textbf{Confidence intervals:} Report 95\% CI via bootstrap; if CI
    width \(> 0.3\), interpret with caution.
  \end{itemize}
\item
  \textbf{Simulation Validation Pending:} We have not yet provided
  simulated evidence that different metacognitive profiles (Socratic,
  Dunning-Kruger, Imposter) are identifiable under realistic noise. This
  is planned for future work.
\item
  \textbf{Analogical Type Theory:} The type-theoretic justification is
  analogical rather than formally constructed. A full domain-theoretic
  or typed lambda-calculus treatment is beyond the current scope.
\item
  \textbf{LLM Operationalization Incomplete:} Applying \(K_n\) to LLMs
  requires addressing question-side shortcuts and model-side signals.
  Specific methods (conformal coverage, debate protocols) are suggested
  but not developed here.
\end{enumerate}

\hypertarget{future-directions}{%
\subsubsection{Future Directions}\label{future-directions}}

\begin{enumerate}
\def\labelenumi{\arabic{enumi}.}
\tightlist
\item
  \textbf{Empirical Validation}: Conduct MAT experiments to validate the
  model's predictions about Socratic wisdom and Dunning-Kruger effect,
  and compare with meta-d' and calibration metrics.
\item
  \textbf{Simulation Studies}: Simulate agents with known \(K\)/\(C\)
  profiles to demonstrate identifiability and estimate required sample
  sizes.
\item
  \textbf{Formal Type Theory}: Develop a typed calculus or algebraic
  data type where \(K\)'s self-application is a well-typed endomorphism.
\item
  \textbf{AI Safety Applications}: Operationalize \(K(K(x))\) for LLMs
  using abstention behavior, self-consistency, and metamorphic testing.
\item
  \textbf{Cultural and Domain Variation}: Investigate whether the
  symmetry assumption (misconception vs.~knowledge) holds across
  cultures and domains.
\end{enumerate}

\hypertarget{why-re-separation-is-unnecessary}{%
\subsubsection{Why R/E Separation is
Unnecessary}\label{why-re-separation-is-unnecessary}}

One alternative formalization might introduce two separate maps: a
subject-level recognition/report map \(R\) and an evaluator/alignment
map \(E\). We argue that such separation is \textbf{unnecessary} for our
framework.

\textbf{The Misunderstanding:}

Such a proposal typically interprets \(K(K(x))\) as: 1. \(K(x)\) =
first-order state (a number) 2. \(K(K(x))\) = applying \(K\) to that
number 3. Therefore \(K(0)\) should equal \(0\)

\textbf{The Reality:}

Our \(K\) is purely \textbf{observational}: 1. \(K_0\) = observation of
State\(_0\) (first-order epistemic state) 2. \(K_1\) = observation of
State\(_1\) (metacognitive state) 3. \(K_2\) = observation of
State\(_2\) (meta-metacognitive state)

\textbf{Each layer observes a different object.} There is no ``subject's
report'' vs ``evaluator's assessment'' --- there is only
\textbf{observation of states}.

\textbf{Why R/E Adds Unnecessary Complexity:}

\begin{longtable}[]{@{}ll@{}}
\toprule\noalign{}
R/E Approach & Our Approach \\
\midrule\noalign{}
\endhead
\bottomrule\noalign{}
\endlastfoot
Introduces ``subject'' as agent & No subject, only states \\
Requires modeling ``perception'' & States exist, observer measures \\
Two functions (R, E) & One function (\(K\)) \\
Subjective/objective split & Purely objective \\
\end{longtable}

\textbf{Our framework is simpler and more elegant because it does not
reify ``the subject'' as a separate entity with ``perceptions.''}

All that exists are \textbf{states} and \textbf{observations}. \(K\)
observes states. That's it.

\hypertarget{references}{%
\subsection{References}\label{references}}

\begin{enumerate}
\def\labelenumi{\arabic{enumi}.}
\tightlist
\item
  Kant, I. (1781). \emph{Critique of Pure Reason}.
\item
  Flavell, J. H. (1979). Metacognition and cognitive monitoring: A new
  area of cognitive-developmental inquiry. \emph{American Psychologist},
  \emph{34}(10), 906-911.
\item
  Nelson, T. O., \& Narens, L. (1990). Metamemory: A theoretical
  framework and new findings. In G. H. Bower (Ed.), \emph{The Psychology
  of Learning and Motivation} (Vol. 26, pp.~125-173). Academic Press.
\item
  Koriat, A. (1993). How do we know that we know? The accessibility
  model of the feeling of knowing. \emph{Psychological Review},
  \emph{100}(4), 609-639.
\item
  Kruger, J., \& Dunning, D. (1999). Unskilled and unaware of it: How
  difficulties in recognizing one's own incompetence lead to inflated
  self-assessments. \emph{Journal of Personality and Social Psychology},
  \emph{77}(6), 1121-1134.
\item
  Maniscalco, B., \& Lau, H. (2012). A signal detection theoretic
  approach for estimating metacognitive sensitivity from confidence
  ratings. \emph{Consciousness and Cognition}, \emph{21}(1), 422-430.
\item
  Shafer, G. (1976). \emph{A Mathematical Theory of Evidence}. Princeton
  University Press.
\item
  Fleming, S. M., \& Daw, N. D. (2017). Self-evaluation of
  decision-making: A general Bayesian framework for metacognitive
  computation. \emph{Psychological Review}, \emph{124}(1), 91-114.
\item
  Dayan, P. (2023). Metacognitive information theory. \emph{bioRxiv}.
  https://doi.org/10.1162/opmi\_a\_00091
\end{enumerate}

\end{document}
