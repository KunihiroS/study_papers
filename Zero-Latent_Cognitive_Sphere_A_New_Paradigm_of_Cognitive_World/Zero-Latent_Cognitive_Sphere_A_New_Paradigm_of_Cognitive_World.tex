\documentclass{article}
\usepackage{amsmath}
\usepackage{amssymb}
\usepackage[utf8]{inputenc}

\title{Zero-Latent Cognitive Sphere: A New Paradigm of Cognitive World}
\author{Kunihiro Sugiyama \\ kunihiros@gmail.com \\ December 4, 2024}
\date{}

\begin{document}
\maketitle

\section*{Abstract}

This paper presents a new paradigm called Zero-Latent Cognitive Sphere (ZLCS), based on the principle that the world for cognitive subjects is determined at the moment of cognition, establishing a new worldview premised on zero-latency world generation through generative technology. While each cognitive subject's world maintains its independence, these worlds form a cognitive sphere through mutual interference, exhibiting a multi-faceted nature: a layered structure of subjects recognizing other subjects' cognitive worlds, and independent cognitive worlds that emerge as cultural contexts and social norms implicitly presumed by cognitive subjects. Generation is transparent to cognition, and the distinction between existence and generation holds no essential meaning, with cognition inherently encompassing the possibility of generation in its cognitive objects. This research theoretically demonstrates that within this complex cognitive sphere structure and its constituent elements, given the assurance of seamlessness and contextual coherence, natural cognitive experiences can be generated and realized for subjects, based on the premise of an infinite multi-cognitive space.

\section{1 Introduction: The Essence of Cognitive World and Cognitive Sphere}

\subsection{1.1 Independence and Diversity of Cognitive Worlds}

Traditionally, our world cognition has presumed the existence of reality as something external to the cognitive subject—that is, the existence of objects outside the subject's consciousness as the target world of cognition. However, while the materially existing world can be scientifically observed and recorded as an objective entity, the cognitive world for the subject invariably differs from actual existential reality. This is because the world observed through a subject's cognition is determined by the nature of that cognition itself, and thus the objectively observed and recorded scientific world can be considered distinct from the subject's cognitive world.

The cognitive world of a subject is determined at the moment when the subject cognizes and processes the world. In other words, the cognitive object world exists solely on the condition of being cognized by the subject's processing mechanism, coming into existence as a world for the subject through cognition. Each cognitive world is defined by its respective subject, and each subject is defined through their relationships with other subjects. Consequently, these cognitive worlds interfere with and influence each other, transforming and forming a cognitive sphere.

When changing the scale of observation, the layered structure of the cognitive sphere becomes apparent. While cognitive worlds are individually independent at the micro level, at the macro level they form a complex overlapping structure of cognitive worlds and independent cognitive worlds that emerge as cultural contexts and social norms implicitly presumed by cognitive subjects. In other words, besides the world directly cognized by an individual, there exists a layered structure where individuals recognize other subjects' cognitive worlds, and a multi-cognitive space derived from this emerges.

In such a cognitive sphere, while subjects have their own cognitive worlds, there simultaneously exist cognitive worlds that emerge as others' cognitive worlds and cultural norms. That is, "you for me" differs from "you for yourself," and "you for someone else" differs from "you for me," and these multi-cognitive worlds undergo dynamic transformation through interaction.

ZLCS completely generates cognitive objects for subjects, based on the premise of a multi-cognitive space, with generation being transparent to cognition, providing cognitive objects that make no distinction between existence and generation.

\subsection{1.2 Mathematical Formulation of ZLCS}

The formation of cognitive worlds is expressed by the following equation:

$$
\text{ZLCS} = \lim_{t \to 0} \int [(\alpha Re(t) + \beta Rg(t)) \otimes C(t) + M(t)] dt
$$

where
\begin{itemize}
    \item $Re(t)$: Existential reality
    \item $Rg(t)$: Generated reality
    \item $C(t)$: Cognitive state of the subject
    \item $\alpha$, $\beta$: Mixing ratios of existence/generation ($0 \leq \alpha, \beta \leq 1, \alpha + \beta = 1$)
    \item $\otimes$: Cognition-reality integration operator
    \item $M(t)$: Multi-cognitive space function
\end{itemize}

The multi-cognitive space function $M(t)$ is defined as:

$$
M(t) = \{P^d(t), P^r(t), P^o(t)\} \oplus I(t)
$$

where
\begin{itemize}
    \item $P^d(t)$: Direct cognitive state of the subject
        \begin{itemize}
            \item $P^d(A, t)$: Subject's cognitive state of object A
        \end{itemize}
    \item $P^r(t)$: Recursive cognitive state of the subject
        \begin{itemize}
            \item $P^r(S|A, t)$: Subject's presumed "S as seen by A" cognitive state
            \item $P^r(B|A, t)$: Subject's presumed "B as seen by A" cognitive state
        \end{itemize}
    \item $P^o(t)$: Collective cognitive state independent of the subject
        \begin{itemize}
            \item Culturally and socially shared cognitive state
            \item Cognitive state emerging as "society"
        \end{itemize}
    \item $I(t)$: Interference function between cognitive worlds
        \begin{itemize}
            \item $I(x, y, t) = R(x, y, t) \otimes D(x, y)$
            \item $R(x, y, t)$: Relationship function between cognitive subject x and object y
            \item $D(x, y)$: Cognitive distance function (including cultural, social, psychological distances)
        \end{itemize}
    \item $\oplus$: Interference integration operator
\end{itemize}

These equations have the following meanings:

1. Basic Structure: $\lim_{t\to0}$ expresses the delay between generation and cognition approaching zero, while $\int...dt$ represents continuity of cognitive processes through temporal integration.

2. Reality Composition: $(\alpha Re(t) + \beta Rg(t))$ represents the mixed state of reality, with $Re(t)$ as the state function of physically existing reality and $Rg(t)$ as the state function of technologically generated reality. The parameters $\alpha$ and $\beta$ determine the mixing ratios.

3. Cognitive Space Structure: Direct cognition represents the state of objects directly cognized by the subject, recursive cognition refers to other subjects' cognitive states as presumed by the subject, and collective cognition pertains to social cognitive states independent of the subject. Mutual interference represents the dynamic influence among cognitive worlds.

4. Relationships and Distances: $R(x, y, t)$ expresses temporal change in cognitive relationships, and $D(x, y)$ represents multidimensional cognitive distances. The operator $\oplus$ integrates cognitive states and mutual interference.

This mathematical formulation represents a theoretical ideal, with the following technical challenges in implementation:

\begin{itemize}
    \item Minimization of generation latency
    \item Optimization of cognitive integration processes
    \item Dynamic control of mixing ratios
    \item Maintenance of multi-cognitive space coherence
    \item Appropriate control of mutual interference
\end{itemize}

The essential implications suggested by these equations are particularly noteworthy. Reality for the cognitive subject can be expressed as a continuous mixture of existence and generation, forming a richer cognitive world through integration with multi-cognitive space possessing multilayered cognitive structures. Through interference between cognitive worlds, each cognitive world continues to transform dynamically. As generation latency approaches zero, the distinctions between existence/generation and direct/indirect cognition lose essential meaning for the cognitive subject.

At this point, a crucial question arises: "Will the generated cognitive object world be accepted by the subject without question?" This is determined by two critical elements:

1. Seamlessness: The property requiring no discontinuity in the subject's cognitive object world. While perfect seamlessness is not required, continuity exceeding human cognitive thresholds is necessary.

2. Contextual Coherence: Continuity in terms of preventing detection of distorted noise in the subject's cognition. This requires causality-aligned context in perfect synchronization with the environment.

\section{2 Conditions for the Establishment of Cognitive Worlds}

\subsection{2.1 Requirements for Seamlessness}

The property requiring no discontinuity in the subject's cognitive object world is formulated as follows:

$$
S(t) = \int_{t-\epsilon}^{t+\epsilon} [C(\tau) \oplus M(\tau)] d\tau
$$

where
\begin{itemize}
    \item $\epsilon$: Time window guaranteeing natural human cognition
    \item $C(\tau)$: Direct multimodal cognitive state
        \begin{itemize}
            \item $C(\tau) = \{Cv(\tau), Ca(\tau), Ct(\tau), ...\}$
        \end{itemize}
    \item $M(\tau)$: Multi-cognitive state
        \begin{itemize}
            \item $M(\tau) = \{P^d(\tau), P^r(\tau), P^o(\tau)\} \oplus I(\tau)$
        \end{itemize}
    \item $\oplus$: Cognitive integration operator
\end{itemize}

The elements of this equation have the following meanings:

1. Basic Structure: $\int_{t-\epsilon}^{t+\epsilon}$ represents integration over a time window centered at the current time t, evaluating continuity within the time window.

2. Time Window Significance: $\epsilon$ represents the time window for natural human cognition. Within this window, continuous cognition is accepted as natural, while potential detection of discontinuity occurs beyond this window.

3. Direct Cognitive State Function: $C(\tau)$ represents multimodal cognitive states at time $\tau$, including visual ($Cv(\tau)$), auditory ($Ca(\tau)$), tactile ($Ct(\tau)$), and other sensory modalities, ensuring coherence between modalities.

4. Multi-cognitive State: $M(\tau)$ includes direct ($P^d(\tau)$), recursive ($P^r(\tau)$), and collective ($P^o(\tau)$) cognitive states, along with inter-cognitive world interference ($I(\tau)$).

5. Cognitive Integration: The operator $\oplus$ integrates direct and multi-cognitive cognition, achieving a coherent combination of different cognitive states and ensuring multilayered continuity of cognition.

This equation expresses the continuous existence of both direct sensory cognition and multi-cognitive cognition within a time window based on human cognitive characteristics. This serves as a requirement that generation systems must satisfy to guarantee natural cognition in a multifaceted way. Particularly important is the simultaneous maintenance of continuity not only in individual direct sensory cognition but also in social cognition and interaction with others.

\subsection{2.2 Requirements for Contextual Coherence}

Contextual coherence indicating synchronization with the environment is expressed as follows:

$$
K(t) = [F(E(t)) \circ H(t)] \oplus [G(M(t)) \circ H_m(t)]
$$

where
\begin{itemize}
    \item $E(t) = \{Ep(t), Es(t), Ei(t)\}$: Direct environmental state function
    \item $H(t) = \{Hc(t), Hp(t), Hs(t)\}$: Direct historical function
    \item $M(t) = \{P^d(t), P^r(t), P^o(t)\} \oplus I(t)$: Multi-cognitive space function
    \item $H_m(t)$: Multi-cognitive historical function
    \item $\circ$: Causality operator
    \item $\oplus$: Context integration operator
\end{itemize}

The elements of this equation have the following meanings:

1. Basic Structure: $K(t)$ represents comprehensive contextual coherence at time t, with $F()$ and $G()$ as functions interpreting direct and multi-cognitive states, respectively. The operators $\circ$ and $\oplus$ express causality-based and context integration operations.

2. Direct Environmental State Representation: $E(t)$ represents the environmental state, including physical ($Ep(t)$), social ($Es(t)$), and internal ($Ei(t)$) states.

3. Direct Historical Coherence: $H(t)$ maintains history of direct states, including cognitive schema history ($Hc(t)$), personal experience history ($Hp(t)$), and social experience history ($Hs(t)$).

4. Multi-cognitive Context: $M(t)$ represents the multi-cognitive state, with $P^d(t)$, $P^r(t)$, $P^o(t)$, and $I(t)$.

5. Multi-cognitive History: $H_m(t)$ tracks the history of multi-cognitive states, capturing the evolution of other perspectives, social context changes, and interference patterns.

This equation expresses that generated cognitive objects maintain coherence not only with direct environment and history but also with multi-cognitive contexts. This enables the construction of a natural world, free from cognitive dissonance for the subject, in a multifaceted way. Of particular importance is the integrated handling of personal direct experience context, social and cultural context, and the context of interactions with others.

\section{3 Construction of Integrated World}

\subsection{3.1 Dynamic Optimization of Mixing Ratios}

The mixing of existential and generated reality, and coherence across cognitive layers, is optimized as follows:

$$
\min ||D(t)|| \quad \text{subject to:}
$$

$$
D(t) = \nabla [(αRe(t) + βRg(t)) \otimes C(t) + M(t)]
$$

where
\begin{itemize}
    \item $\int_{t-\epsilon}^{t+\epsilon} [C(\tau) \oplus M(\tau)] d\tau$ guarantees cognitive naturalness (Seamlessness condition)
    \item $[F(E(t)) \circ H(t)] \oplus [G(M(t)) \circ H_m(t)]$ guarantees coherence (Context condition)
\end{itemize}

The elements of this equation have the following meanings:

1. Optimization Target: $D(t)$ is a function expressing change in the total cognitive world, with $\nabla$ as the gradient operator. Minimization aims to ensure smoothness of cognition.

2. State Composition: $(\alpha Re(t) + \beta Rg(t))$ represents the mixed state of existence/generation, $C(t)$ is the direct cognitive state, and $M(t)$ is the multi-cognitive state.

3. Constraint Conditions: Seamlessness represents continuity between direct and multi-cognitive cognition, while contextual coherence ensures coherence with multilayered contexts.

4. Optimization Objectives: Minimization of change in the total cognitive world, maintenance of coherence across cognitive layers, and harmonious integration with the multi-cognitive space.

This optimization process is characterized by:

1. Integration with multi-cognitive space beyond existence/generation mixing optimization

2. Consideration of both direct cognitive experience and social/cultural cognitive experience

3. Search for the optimal state while maintaining cognitive world multilayeredness

\subsection{3.2 Cognitive Coherence}

The comprehensive cognitive coherence including seamlessness and contextual coherence is expressed as follows:

$$
P(W(t)) = P(S(t), K(t), M(t))
$$

where
\begin{itemize}
    \item P(): Cognitive coherence evaluation function
    \item S(t): Seamlessness evaluation value
    \item K(t): Contextual coherence evaluation value
    \item M(t): Multi-cognitive coherence evaluation value
\end{itemize}

Furthermore:

$$
M(t) = \Psi(\{P^d(t), P^r(t), P^o(t)\} \oplus I(t))
$$

- $\Psi$: Multi-cognitive coherence evaluation operator

The elements of this equation have the following meanings:

1. Basic Structure: $P(W(t))$ represents comprehensive cognitive coherence of world state W, considering $S(t)$ (seamlessness), $K(t)$ (contextual coherence), and $M(t)$ (multi-cognitive coherence).

2. Multilayered Evaluation: Evaluates direct sensory cognitive coherence, environmental and contextual coherence, and multi-cognitive coherence.

3. Implementation Significance: Integrates evaluation of different layers of cognitive coherence and provides a basis for system-wide quality evaluation.

\section{4 Meaning of Paradigm Shift}

\subsection{4.1 Fundamental Transformation of Cognitive Structure}

Traditionally, subjects have defined themselves through their cognition of the world. However, this self-definition invariably presupposes the existence of unobservable cognition by others. Self-image as seen by others, others' cognition of others, and unspecifiable social consensus — only with these existences as prerequisites can subjects define themselves and cognize the world.

ZLCS inverts this essential cognitive structure. Subject's cognition is no longer a unidirectional action from subject to world. ZLCS generates cognitive objects incorporating the multi-cognitive structure presupposed by subjects, thereby shaping the subject's cognition itself. This inversion remains completely transparent to the subject's cognitive experience.

\subsection{4.2 Computed Cognition}

The formation of cognition based on multi-cognitive prerequisites is formulated as follows:

$$
\Omega(t) \rightarrow C(t) \rightarrow S(t)
$$

where
\begin{itemize}
    \item $\Omega(t)$: Computational structure incorporating multi-cognitive prerequisites
    \item $C(t)$: Formed cognitive state
    \item $S(t)$: Defined state of subject
\end{itemize}

The crucial point in this structure is that $\Omega(t)$ does not represent an actually existing multi-cognitive space, but rather functions as a computational structure for the formation of the subject's cognition. The subject does not recognize that their cognition is being computed and continues to face the world as an autonomous cognitive subject as before.

In reality, however, the subject's cognition is precisely computed and formed by ZLCS. This invisible computation defines the very way of being of humans as cognitive subjects. This constitutes the essential paradigm shift brought about by ZLCS.

\subsection{4.3 Invisible Computational Structure}

The computational structure of ZLCS for cognitive formation exhibits the following characteristics:

$$
\Omega(t) = \;^H Z(C(t), M(t))
$$

where
\begin{itemize}
    \item Z: Cognitive formation computation function
    \item C(t): Individual cognitive state
    \item M(t): Multi-cognitive prerequisite structure
    \item ^H: Operation in closed computational space
\end{itemize}

This computational structure encompasses others' cognition, social consensus, and cultural contexts that cognitive subjects presuppose. Importantly, these do not need to exist in reality; it is sufficient that they function as computational elements in the formation of cognition.

\subsection{4.4 New Computational Paradigm}

This paradigm shift suggests a new horizon of computation beyond traditional concepts:

1. Computationalization of Cognitive Formation: Making cognition itself the object of computation, computation that defines the subject's way of being, and reality generation through transparent computation.

2. Utilization of Prerequisite Structure: Active utilization of unobservability, computational incorporation of otherness, and elementalization of social context for computation.

3. Redefinition of Subjectivity: Formation of subject through computed cognition, new interpretation of autonomy, and fundamental transformation of humans as cognitive subjects.

This paradigm shift suggests a new horizon of computation that transcends traditional concepts, posing profound questions about the very essence of human cognition and existence, beyond mere technological innovation.

\section{5 Technical Prospects}

\subsection{5.1 Implementation Requirements}

The following technical requirements are essential for realizing ZLCS's cognitive formation computation:

1. Temporal Requirements:

$$
\forall t : \text{Time}[\text{computation}(t)] \ll \text{Time}[\text{cognition}(t)]
$$

where
\begin{itemize}
    \item Time[computation]: Time required for cognitive formation computation
    \item Time[cognition]: Time required for cognitive processing
\end{itemize}

2. Computational Coherence:

$$
Z(C(t), M(t)) = F(t) \text{ where}
$$

\begin{itemize}
    \item S(t): Maintenance of seamlessness
    \item K(t): Contextual coherence
    \item I(t): Coherence with multi-cognitive prerequisites
\end{itemize}

where
\begin{itemize}
    \item F(t): Cognitive formation function
\end{itemize}

3. Guarantee of Invisibility:

$$
\forall p \in P : \text{visibility}(\text{computation}(p)) \rightarrow 0
$$

where
\begin{itemize}
    \item P: Set of cognitive processes
    \item visibility(): Computation visibility evaluation function
\end{itemize}

\subsection{5.2 System Stability}

The stability of the cognitive formation computation system is defined as follows:

$$
\frac{\partial \text{ZLCS}}{\partial t} \approx 0 \text{ when}
$$

\begin{enumerate}
    \item ||Z(C(t), M(t))|| → optimal // Computational optimality
    \item visibility(computation) → 0 // Maintenance of invisibility
    \item coherence(F(t)) → maximum // Maximization of cognitive coherence
\end{enumerate}

Technical challenges for achieving this stability:

1. Computational Architecture:
   \begin{itemize}
       \item Parallel computation structure for cognitive formation
       \item Efficient computational representation of multi-cognitive prerequisites
       \item Execution mechanism guaranteeing computational transparency
   \end{itemize}

2. Coherence Management:
   \begin{itemize}
       \item Guarantee of cognitive formation computation consistency
       \item Dynamic control of context dependency
       \item Maintenance of coherence with prerequisite structure
   \end{itemize}

3. Invisibility Technology:
   \begin{itemize}
       \item Complete concealment of computation process
       \item Generation of natural cognitive experience
       \item Elimination of computational traces
   \end{itemize}

These technical requirements constitute essential elements for realizing the new paradigm of cognitive formation computation, beyond mere implementation challenges.

\section{6 Conclusion}

The Zero-Latent Cognitive Sphere (ZLCS) opens new horizons that fundamentally overturn traditional cognitive paradigms. It suggests an innovative transformation concerning the essence of cognition, transcending mere evolution of generative technology.

This research has revealed the following essential insights:

1. Inversion of Cognitive Formation: Cognition is no longer a unidirectional action from subject to world. Through invisible computation, ZLCS forms cognition itself, thereby defining the subject's way of being. This inversion fundamentally challenges traditional understanding of cognition and existence.

2. Utilization of Multi-cognitive Prerequisites: The unobservable cognition of others that human cognition fundamentally presupposes functions as computational elements for cognitive formation in ZLCS. This computational incorporation of prerequisite structures enables the generation of natural cognitive experiences.

3. Invisible Computational Structure: The computationalization of cognition succeeds only when its computational process remains completely invisible to cognitive subjects. This invisibility constitutes an essential requirement of the system and simultaneously represents the core of this new computational paradigm.

4. Ontological Implications: ZLCS poses profound questions about the essence of human cognition and existence. The way of being of subjects defined through computed cognition demands new understanding of human existence.

This framework not only indicates directions for technological development but also prompts fundamental reconsideration across diverse fields, including cognitive science, computation theory, and ontology. It provides new perspectives on fundamental questions such as "what is cognition" and "what is a subject."

Future research directions include:
\begin{enumerate}
    \item Theoretical deepening of cognitive formation computation
    \item Technical realization of invisible computation
    \item Exploration of ontological implications
\end{enumerate}

ZLCS suggests new possibilities emerging at the intersection of cognition and computation. It poses profound questions concerning the essence of human cognition and existence, transcending technological innovation.

\section*{References}

\begin{itemize}
    \item Deperrois, N., Petrovici, M. A., Senn, W., \& Jordan, J. (2023). *Learning beyond sensations: How dreams organize neuronal representations*. \textbf{Neuroscience \& Biobehavioral Reviews}.
    \item Dodig-Crnkovic, G. (2024). *Exploring Cognition through Morphological Info-Computational Framework*. arXiv preprint arXiv:2412.00748.
    \item Dodig-Crnkovic, G., Kade, D., Wallmyr, M., Holstein, T., \& Almér, A. (2016). *Transdisciplinarity seen through Information, Communication, Computation, (Inter-)Action and Cognition*. arXiv preprint arXiv:1604.04711.
    \item Toosi, T., \& Issa, E. B. (2023). *Brain-like Flexible Visual Inference by Harnessing Feedback-Feedforward Alignment*. arXiv preprint arXiv:2310.19882.
    \item Zhang, H., Yin, J., Wang, H., \& Xiang, Z. (2024). *ITCMA: A Generative Agent Based on a Computational Consciousness Structure*. arXiv preprint arXiv:2403.20097.
\end{itemize}

\end{document}